\begin{thebibliography}{99}

\bibitem{IDS}
S. Mukkamala, A.H. Sung, \textit{A Comparative Study of Techniques for Intrusion Detection} kirjassa Tools with Artificial Intelligence, 
15th IEEE International Conference, 2003.

\bibitem{WEBS}
B. Shweta, \textit{Web Security Basics}, Course Technology, 2002.

\bibitem{DDOS}
D. Dittrich, S. Dietrich, J. Mirkovic, P. Reiher, \textit{Internet Denial of Service: Attack and Defence Mechanisms}, Prentice Hall PTR, 2004.

\bibitem{DDOSb}
S. Ranjan, R. Swaminathan, M. Uysal, E. Knightly, \textit{DDoS-Resilient Scheduling to Counter Application Layer Attacks under Imperfect Detection},
Infocom 2006, 25th IEEE International Conference on Computer Communications, 2006.

\bibitem{FBI}
FBI, \textit{Wanted by the FBI - Saad Echouafni}, <URL: \texttt{http://www.fbi.gov/wanted/fugitives/cyber/echouafni\_s.htm}>, viitattu 27.10.2009.

\bibitem{CERT}
CERT, \textit{Denial of Servce}, saatavilla WWW-muodossa <URL: \texttt{http://www.cert.org/tech\_tips/denial\_of\_service.html}>, viitattu 20.10.2009.

\bibitem{STACK}
R. Bandes, B. Franklin, M. Gregg, G. Mays, C. Ries, S. Watkins, \textit{Hack the Stack}, Syngress Publishing Inc, 2006. 

\bibitem{IDSb}
R. Lippmann, D. Stetson, S. Webster, \textit{The Effect of Identifying Vulnerabilities and Patching Software on the Utility of Network Intrusion Detection},
kirjassa Recent Advances in Intrusion Detection, 2002.

\bibitem{TCP}
K. Kaario, \textit{TCP/IP-verkot}, Docendon, 2001.

\bibitem{CVE}
CVE, <URL: \texttt{http://www.cve.mitre.org/cve}>, viitattu 29.10.2009.

\bibitem{SYM}
Symantec, \textit{Symantec Internet Security Threat report. Trends for July-December 2007}, saatavilla WWW-muodossa
<URL: \texttt{http://www.symantec.com/business/index.jsp}>, viitattu 1.12.2009.

\bibitem{SYM2}
Symantec, \textit{Symantec Global Internet Security Threat Report. Trends for 2008}, saatavilla WWW-muodossa <URL: \texttt{http://www.symantec.com/business/index.jsp}>,
viitattu 1.12.2009.


\bibitem{U2R}
Z. Bankovic, S. Bojanic, O. Nieto-Taladriz, A. Badii, \textit{Increasing Detection Rate of User-to-Root Attacks Using Genetic Algorithms}, 
Emerging Security Information, Systems, and Technologies, 2007.

\bibitem{SQL SS}
C. Andrews, D. Litchfield, \textit{SQL Server Security}, McGraw-Hill Osborne, 2003.

\bibitem{WEB2}
R. Cannings, H. Dwidedi, Z. Lackey, \textit{Hacking Exposed Web 2.0: Web 2.0 Security Secrets and Solutions}, McGraw-Hill Companies, 2008.

\bibitem{WEB2b}
S. Shreeraj, \textit{Web 2.0 Security: Defending Ajax, RIA and SOA}, Course Technology, 2007.

\bibitem{WEB2c}
G. Lawton, \textit{Web 2.0 Creates Security Challenges}, Computer-IEEE Computer Society, saatavilla WWW-muodossa <URL: \texttt{http://www.computer.org}>,
viitattu 5.11.2009.

\bibitem{SOA}
IBM Redbooks, \textit{Patters: Service-Oriented Architecture and Web Services}, IBM, 2004.

\bibitem{Hacking}
S. McClure, J. Scambray, G. Kurtz, \textit{Hacking Exposed (5th Edition)}, McGraw-Hill Osborne, 2005.

\bibitem{XSS}
XXSed, <URL:\texttt{http://http://www.xssed.com/archive/special=1}>, viitattu 2.12.2009.

\bibitem{CSRF}
A. Barth, C. Jackson, J.C. Mitchell, \textit{Robust Defences for Cross-Site Request Forgery}, Conference on Computer and Communication Security, 
Proceedings of the 15th ACM conference, 2008.

\bibitem{CSRFb}
Z. Mao, N. Li, I. Molloy, \textit{Defeating Cross-Site Request Forgery Attacks with Browser-Enforced Authenticity Protection}, Lecture Notes in Computer Science,
Springer Berlin / Heidelberg, 2009.

\bibitem{WASC}
Web Application Security Consortium, <URL:\texttt{http://www.webappsec.org}>, viitattu 9.12.2009.

\bibitem{WASCb}
Web Application Security Consortium / Web Application Security Statistics, <URL:\texttt{http://projects.webappsec.org/Web-Application-Security-Statistics}>, 
viitattu 9.12.2009.

\bibitem{Nikto}
Nikto, saatavilla osoitteesta <\texttt{http://www.cirt.net/nikto2}>, viitattu 9.12.2009.

\bibitem{INS}
Insecure.org, Top 10 Web Vulnerability Scanners, <\texttt{http://sectools.org/web-scanners.html}>, viitattu 9.12.2009.

\end{thebibliography}
