% -*- mode: LaTeX; coding: utf-8; -*-

\chapter{Aiheeseen liittyvä tutkimus}

\section{Yleistä}

Viime vuosina web-sovellukset ovat kasvattaneet suuresti suosiota, ja nykyisin yhä useampi palvelu, jossa tietoturvan ja saatavuuden takaaminen on elintärkeää, on siirtynyt osittain tai kokonaan 
verkon puolelle. Tämä on vaikuttanut suuresti siihen, mihin tietoturvahyökkäykset nykyisin kohdistuvat ja kuinka niitä pyritään toteuttamaan. Hyökkäysten muuttuessa yhä hienostuneimmiksi, 
eivät perinteiset tietoturvamenetelmät enää riitä suojaamaan loppukäyttäjiä tai palvelun ylläpitäjiä. Tästä syystä erilaisten tietoturvaratkaisuiden ympärillä käy kova kuhina, ja aihepiiri 
onkin herättänyt suurta kiinnostusta tutkijoiden keskuudessa. Erilaisia ratkaisuja, joissa on pyritty selvittämään aikaisemmissa luvuissa esitettyjä tietoturvahyökkäyksiä ja näiden 
mukana tulleita haasteita, on lukematon määrä. Seuraavaksi esitellyt tutkimukset edustavat vain pientä osaa koko tutkimuskentästä, mutta näistä käy kuitenkin jo ilmi se, millaisiin ratkaisumalleihin
nykyisin pyritään.

Hyökkäysten tunnistaminen toimii siten, että analysoimalla yhtä tai useampaa tapahtumaa pyritään löytämään viitteitä tapahtuvista hyökkäyksistä. Tunnistamiseen pohjautuvat menetelmät jaetaan usein
kahteen eri tyyppiin: anomalioiden eli poikkeavuuksien tunnistamiseen (eng. anomaly detection) ja väärinkäytösten tunnistamiseen (eng. misuse detection). Näistä anomalioiden tunnistamiseen perustuvat
järjestelmät luovat malleja järjestelmän, käyttäjän tai verkon normaalista käyttäytymisestä, ja vertaamalla tapahtumia näin muodostettuihin kuvauksiin voidaan poikkeavuudet tunnistaa. Väärinkäytösten 
tunnistamiseen tarkoitetut järjestelmät taas sisältävät joukon kuvauksia eli signatuureja tunnetuista hyökkäyksistä. Tuleva liikenne tarkistetaan näitä kuvauksia vastaan, jolloin kuvauksia vastaavat
hyökkäykset voidaan tunnistaa. Joissakin tapauksissa jaottelu on myös tehty sen mukaan, mistä tutkittava liikenne on kerätty. Tällöin järjestelmät on jaettu verkkoon pohjautuviksi (eng. network based)
ja asiakaspohjaisiksi (eng. host based). 

\section{Väärinkäytösten tunnistaminen}

Väärinkäytökseen perustuvat järjestelmät ovat pitkään olleet suosituin lähestymistapa hyökkäysten torjumisessa. Nämä järjestelmät on voitu jakaa kahteen erilliseen osaan, jossa tilattomassa 
järjestelmässä jokaista tulevaa tapahtumaa käsitellään itsenäisesti, kun taas tilallisessa järjestelmässä tutkitaan tapahtumien välisiä suhteita. Web-pohjaisten hyökkäysten tutkimisessa tietolähteinä 
on käytetty mm. palvelimien tuottamaa lokia \cite{LightTool} ja IDS-järjestelmien \cite{Snort}\cite{Bro} tapauksessa analysoimalla verkkokerroksen liikennettä. Ensimmäisessä ratkaisussa ongelmana
on, että itse palvelimiin kohdistuvia hyökkäyksiä ei pystytä havaitsemaan ainoastaan lokitietoa tarkkailemalla. Samoin hyökkäyksiä, jotka koostuvat monista eri vaiheesta, on mahdotonta mallintaa. 
Verkkokerroksella toimivia järjestelmiä taas pystytään harhauttamaan muokkaamalla hyökkäyksiä, ja näistä harvat mahdollistavat tilallisen analyysin. 

Esitettyjä ongelmia on pyritty ratkaisemaan lisäämällä havaittavien tapahtumien määrää ja keräämällä informaatiota eri lähteistä. Esimerkiksi asentamalla sovellustasolle erillinen tiedonkeruuseen 
tarkoitettu komponentti \cite{Application} on saatu hyviä tuloksia. Tässä tapauksessa tiedonkeruu tapahtui Apache-palvelimelle asennetun komponentin välityksellä, joka tarkkaile lokitiedon lisäksi
mm. pyyntöjen tulkitsemiseen ja toteuttamiseen menevää aikaa. Tämän ratkaisun etuna on myös se, että tutkittava data on salaamattomassa muodossa ja sessioiden uudelleenrakentaminen on mahdollista. 
Tällä tasolla toimiva IDS-järjestelmä voi myös toimia ennaltaehkäisevästi eli haitalliseksi havaittu liikenne voidaan tiputtaa pois ennen kuin se käsitellään palvelimella. Suurimpana haittapuolena on, 
että tietylle sovellukselle suunniteltua komponenttia ei voida sellaisenaan käyttää muilla alustoilla. 

WebSTAT \cite{Webstat} on tilalliseen analyysiin perustuva IDS-järjestelmä, joka pohjautuu STAT-kehitysympäristöön \cite{STAT}. WebSTAT hyödyntää STATL-ohjelmointikieltä, joka mahdollistaa hyökkäysten
mallintamisen ottamalla huomioon mm. eri tapahtumien välisiä yhteyksiä ja verkkohistoriaa sekä palvelimien kuten Apachen ja Microsoftin IIS:n lokia. Koska tietoa kerätään yhtäaikaisesti monesta eri 
lähteestä, voidaan palvelimien lokitiedon analysointiin yhdistää alempien toimintojen kuten jär\-jes\-tel\-mä- ja verkkotason tuottamaa tietoa. Näin tehty analyysi kuvaa koko järjestelmä tilaa, jolloin 
yllättävät muutokset pystytään havaitsemaan nopeasti. Testien mukaan tällainen analyysi voidaan toteuttaa suurissa järjestelmissä reaaliajassa aiheuttamatta suurempaa viivettä palvelimien toiminnassa. 
Menetelmän sovittaminen tiettyyn järjestelmään vaatii jonkin verran manuaalista työtä, ja tätä voidaankin pitää sen suurimpana heikkoutena. Monimutkaisten hyökkäysten kuvaaminen on usein myös hankala
toteuttaa, ja niiden tulkitsemiseen saattaa tuhlaantua turhaa aikaa.

Väärinkäytösten tunnistamiseen tarkoitettuja järjestelmiä on hyödynnetty myös uudempien tietoturvahyökkäysten tunnistamisessa. Esimerkiksi vihamielisten Flash-mainosten tunnistamiseen tarkoitettu 
OdoSwiff \cite{FlashAdd} pyrkii etsimään sivuilla olevista mainoksista hyökkäykseen tarkoitettua koodia käyttäen staattista ja dynaamista analyysia. Palvelinpuolella XSS-hyökkäyksiä vastaan on luotu 
järjestelmä, josta löytyy yleisimpien hyökkäysten kuvaukset \cite{SignatureXSS}. Kumpikin järjestelmistä toimii erittäin hyvin, kunhan hyökkäys on entuudestaan tuttu.

\section{Poikkeavuuksien tunnistaminen}

Väärinkäytösten tunnistamiseen käytettyjen järjestelmien suurin heikkous piilee siinä, että mallintamattomat hyökkäykset jäävät näiltä huomaamatta. Tämä on erityisesti web-palveluihin kohdistuvien
hyökkäysten tapauksessa iso ongelma, sillä toimintaympäristö muuttuu jatkuvasti, ja uusia hyökkäyksiä ja vanhojena variaatioita ilmestyy tiuhaan tahtiin. Tällöin signatuurien pitäminen ajantasalla
muodostuu mahdottomaksi tehtäväksi. Hyökkäysten monimuotoisuus onkin johtanut siihen, että nykyisin yhä useammassa järjestelmässä pyritään tunnistamaan poikkeavat tapahtumat normaaliin liikenteen seasta
ilman tarkkoja signatuureja. 

Anomalioiden tunnistamiseen on käytössä useita eri menetelmiä, ja ne voidaan jakaa kahteen eri ryhmään \cite{State}. Näistä ensimmäinen sisältää oppimiseen pohjautuvat mallit, jossa normaali 
käyttäytyminen opetetaan opetusmateriaalin avulla. Normaalia käyttäytymistä esittävät profiilit voidaan mallintaa käyttäen joko sään\-tö\-poh\-jais\-ta-, mallipohjaista- tai tilastopohjaista menetelmää. Näistä
sääntöpohjaiset menetelmät muistuttavat eniten perinteisiä IDS-järjestelmiä sillä erolla, että luodut säännöt pohjautuvat kerättävään dataan, ja säännöt voivat olla rakenteiltaan hyvin monimutkaisia. 
Mallipohjaisissa menetelmissä taas luodaan normaalia käyttäytymistä kuvaavat profiilit, jota vastaan tuleva liikenne arvioidaan. Tiedonlouhinta, neuroverkot ja liikenteestä luotujen kuvioiden vertaaminen
tulevaan liikenteeseen (eng. pattern matching) ovat tekniikoita, joita on käytetty tällaisten mallien luomiseen. Analyysissa voidaan esimerkiksi tutkia verkkopakettien kuormia \cite{Payload}\cite{ULISSE} 
ja klusteroimalla ja luokittelemalla liikenne protokollien ja palveluiden mukaan \cite{Cluster}. Viimeisen ryhmän muodostavat tilastollisiin menetelmiin pohjautuvat menetelmät \cite{PacketHeader}, jotka
ovat jääneet vähemmälle huomiolle johtuen alati muuttuvasta toimintaympäristöstä.

Toisen ryhmä anomalioiden tutkimisessa muodostavat spesifistiset mallit (eng. spesification model). Nämä menetelmät pohjautuvat enemmän ihmisten huomioihin ja asiantuntijuuteen kuin matemaattisiin
kuvauksiin. Menetelmissä käytetään useita eri elementtejä aina sovellustasolta verkkotasolle, ja näitä käyttäen luodaan normaalia käyttäytymistä kuvaavat mallit. Järjestelmät voivat hyödyntää esimerkiksi
protokollista kerättävää tietoa anomalioiden tunnistamisessa. Järjestelmien eri tiloja ja tapahtumien välisiä suhteita voidaan myös mallintaa, jolloin poikkeavat tilat ja tapahtumaketjut voidaan
tunnistaa. 

Anomalioiden tunnistamiseen perustuville järjestelmille löytyy useita eri käyttökohteita. Niillä voidaan esimerkiksi pyrkiä tunnistamaan tietokantoihin kohdistuvia tunnettuja ja tuntemattomia SQL-hyökkäyksiä.
Järjestelmälle voidaan opettaa normaali käyttäytyminen esimerkiksi tiedonlouhintamenetelmin \cite{Data} tai luomalla profiileja normaalista käyttäytymisestä \cite{SQLanomaly}\cite{SQLlearning}. Erilaisia
menetelmiä voidaan myös yhdistellä, jolloin todennäköisyys poikkeavan liikenteen tunnistamiseen kasvaa. Tutkimalla esimerkiksi yhtä aikaisesti sekä web-pyynnöissä että SQL-kyselyissä ilmeneviä poikkeavuuksia, 
voidaan tulevat kyselyt pisteyttää tarkasti \cite{WebSQL}. Kyselyiden kategoriointi mahdollistaa sen, että haitalliseksikin merkityt kyselyt voidaan ohjata sellaisille palvelimille, joilla ei ole 
pääsyä arkaluontoiseen tietoon. 

Web-palveluihin kohdistuvien hyökkäysten tunnistaminen on hankala ja aikaa vievä prosessi. Aikaisemmin tässä työssä esitetyt hyökkäykset kattavat vain osan hyökkäyksistä, joita vastaan sovellussuunnittelijat ja
ylläpitäjät joutuvat suojautumaan. Juuri tämä hyökkäysten monimuotoisuus on se seikka, joka on nostanut anomaliatutkimuksen muiden menetelmien yläpuolelle, ja aihepiiri on viime vuosina noussut yhdeksi 
puhutuimmista tietoturvan saralla. 

Poikkeavan tilan tunnistamiseen on käytetty monia eri menetelmiä, ja päätöksen tekemiseen on käytetty useita eri tietolähteitä. Esimerkiksi Swaddler \cite{Swaddler} on 
web-sovelluksille suunniteltu menetelmä, joka oppii kriittisten järjestelmäkutsujen ja sovelluksen tilojen väliset suhteen analysoimalla web-sovelluksen sisäisiä tiloja. Tällä tavoin voidaan tunnistaa hyökkäykset,
jotka aiheuttavat poikkeavia tiloja esimerkiksi sovelluksen normaaliin työkulkuun. Järjestelmä koostuu eri malleista ja osista, joille opetetaan harjoittelujakson aikana sitä vastaava normaali käytös. Jokaiselle
osalle, jotka käytännössä vastaavat tiettyjä sovelluksen toimintoja, lasketaan myös kynnysarvo, jonka ylittyessä sen aiheuttanut toiminto lasketaan anomaliaksi. Tehdyissä testeissä järjestelmä tunnisti 
kaikki toteutetut hyökkäykset, ja virheellisten positiivisten hälystysten määrä oli kohtuullisen pieni. Analysointi aiheutti jonkin verran kuormaa palvelimelle, mutta optimoimalla toteutusta tämä voidaan poistaa
lähes kokonaan. 

Aikaisemmin esitettyä tapaa, jossa hyödynnetään palvelimen tuottamaa logia, voidaan käyttää myös poikkeavuuksien tunnistamisessa \cite{Multi}. Tässä tapauksessa analysoitiin Apachen tuottamaa HTTP-logia, ja huomio kiinnitettiin kyselyihin, joissa parametreja käyttäen välitettiin arvoja palvelinpuolen ohjelmille tai aktivoitiin dokumentteja. Kyselyt purettiin osiin, ja niitä analysoitiin käyttäen useita eri malleja 
mm. kyselyjen pituutta ja normaalia järjestystä, merkkien jakaumia ja pyyntöjen tiheyttä. Vastaavaa mallia on sovellettu myös poikkeavien järjestelmäkutsujen tunnistamisessa \cite{SystemCall}, joten sen käyttö ei rajoitu pelkästään web-palveluihin. Väärinkäytös- ja anomaliamenetelmien yhdistämistä on myös ehdotettu \cite{Combination}. Tällainen järjestelmä voi toimia siten, että tuleva liikenne syötetään ensiksi anomalioita tutkivalle järjestelmälle, ja vain tunnistetut positiiviset tapaukset ohjataan väärinkäytösjärjestelmälle. Menetelmän etuna on, että virheellisten positiivisten määrä tippuu suuresti, ja tarkan analyysin vaativat tapahtumat pienenevät.


