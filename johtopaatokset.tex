% -*- mode: LaTeX; coding: utf-8; -*-

\chapter{Yhteenveto}

Monien perinteisten toimintojen siirtyessä kohti digitaalisia palvelumalleja, on erilaisten 
Web-sovellusten ja -palveluiden merkitys kasvanut huomattavasti. Lisäksi yhä useampi näistä sovelluksista
toimii tietoturvan kannalta kriittisillä sektoreilla.
Tämän muutoksen ovat havainneet myös tietomurtoja tekevät hyökkääjät, jotka pyrkivät kaikin
keinoin hyödyntämään järjestelmistä löytyviä heikkouksia rikollisiin tarkoituksiin.

Tietoturva on aina kulkenut alan muuta kehitystä hiukan jäljessä, ja tätä seikkaa hyökkääjät
ovat aina käyttäneet hyväksi. Nykytilanne noudattaa myös samaa kaavaa. Perinteisiltä
tietoturvahyökkäyksiltä osataan jo kohtuullisesti suojautua ja löydetyt haavoittuvuudet
korjataaan riittävän nopeasti. Sovellusten tekijät ja palveluiden tarjoajavat tuntevat
myös sen verran hyvin oman toimintaympäristönsä, että suuria virheitä ei pääse syntymään.

Web-pohjaisissa palveluissa tilanne on toinen. Aihealue on vielä sen verran tuore,
että sovellusten kehittäjät ja ylläpitäjät yrittävät vasta sopeutua siirroksen
tuomiin muutoksiin. Uusien teknologioiden ja kehitysympäristöjen oppimiseen
menee aina oma aikansa, ja näiden turvalliseen käyttöön vielä pidempään. 
Tähän kun lisätään mukaan uudenlaiset arkkitehtuurimallit, joihin kehittäjät yrittävät 
sopeutua, niin on varmaa, että sovelluksiin eksyy erilaisia haavoittuvaisuuksia.

Tähän muutoksen tuomaan epävarmuuteen tietomurtoja ja -varkauksia tekevät hyökkääjät ovat iskeneet. 
Erilaisia uusia hyökkäystapoja kehitetään jatkuvasti ja väärinkäytettyjä haavoittuvaisuuksia löydetään lähes päivittäin.
Monet näistä hyökkäyksistä eivät edes pyri ohittamaan asetettuja suojauksia, vaan ne käyttävät hyödykseen
sovelluksen omia toimintoja, joita ei vain ole osattu riittävästi suojata. Osa näistä heikkouksista on 
myös niin syvällä käytetyssä tekniikoissa tai alustoissa, että suoraan niitä ei pystytä edes korjaamaan vaan
korjaaminen vaatii kokonaan uudenlaista lähestymistapaa. Tämä ajaakin monet suunnittelijoista tekemään
omia pikaisia korjauksia, joiden toimivuus on usein kuitenkin hyvin tapauskohtaista. 

Uudenlaisten tietoturvahyökkäyksten tunnistaminen vanhoilla työkaluilla on usein hyvin hankalaa tai
lähes mahdotonta. Yhteistä erilaisille hyökkäyksille on, että ne poikkeavat yleensä jollakin tavoin normaalista
tietomassasta. Ongelmana on se, kuinka nämä poikkeavuudet eli anomaliat tunnistetaan suuresta tietomassasta
riittävällä nopeudella ja tarkkuudella. Tätä tunnistamista varten onkin kehitetty useita erilaisia menetelmiä,
jotka pyrkivät jollakin tavoin erottamaan hyökkäykset muusta liikenteestä. 

Tämän tutkielman tavoitteena oli tunnistaa poikkeavat HTTP-kyselyt automaattisesti suuresta tietomassasta.
Käytettäväksi menetelmäksi valittiin diffuusiokuvaukset, jotka tarjoavat nopean ja tehokkaan tavan tunnistaa poikkeavuudet
järjestelmän normaalista käytöksestä. Materiaalin käsittelyä varten kehitettiin erillinen sovellus, joka suoritti 
lähdemateriaalille halutut toiminnot. Varsinainen analyysi suoritettiin Matlab-sovelluksessa, jonka jälkeen
poikkeaviksi merkityt pisteet käytiin yksitellen läpi.

Saadut tulokset osoittavat, että diffuusiokuvausmenetelmä soveltuu hyvin tämän tyyppisen materiaalin analysointiin.
Poikkeaviksi havaitut kyselyt erosivat myös todellisuudessa normaalista tietomassasta. Tulokset eivät kuitenkaan osoita sitä,
että diffuusiokuvaus olisi paras valinta käytännön sovelluksiin. Tätä varten tarvitaan vielä paljon jatkotutkimusta ja vertailua
muihin menetelmiin, johon ei tässä työssä perehdytty ollenkaan. Alustavat tulokset antavat kuitenkin uskoa siihen, että
tulevaisuudessa diffuusiokuvausten käyttö on mahdollista eri sovellusalueilla.



