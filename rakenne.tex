% MASTER'S THESIS IN MATHEMATICAL INFORMATION TECHNOLOGY
% Tuomo Sipola
%
% Based on Timo Männikkö's template. 
% Uses latex template by Antti-Juhani Kaijanaho and Matthieu Weber.
%

\documentclass[finnish,logo,nonumbib,nocopyright,lof,pdftex]{gradu2}

\usepackage[utf8]{inputenc}

\usepackage{palatino} % Better font
\usepackage[intlimits]{amsmath} % For better math mode, namely integrals
\usepackage{amssymb} % Math symbols

\usepackage[pdftex]{graphicx} % Pictures
\usepackage{pdfpages} % For inclusion of the article

% Optional 1,5 point line spacing
%\usepackage{setspace}
%\onehalfspace

% Define our own abbreviations list
\newenvironment{abbrlist}[1]{
  \begin{list}{}{\settowidth{\labelwidth}{\textbf{#1}}
  \setlength{\leftmargin}{\labelwidth}
  \addtolength{\leftmargin}{\labelsep}
  \renewcommand{\makelabel}[1]{\textbf{\hfill##1}}}}%
{\end{list}}

% No spacing with this enumeration
\newenvironment{enumerate_no_space}{
  \begin{enumerate}
  \setlength{\itemsep}{1pt}
  \setlength{\parskip}{0pt}
  \setlength{\parsep}{0pt}}
{\end{enumerate}}

% Remoge pagebackref in the final version
% Add "pagebackref=true" to get page numbers where each reference is used
% Link version for internet viewing
%\usepackage[pdftex,colorlinks=true]{hyperref}

% PDF information tags
\pdfinfo{/Title (Applying Hilbert-Huang Transform to Mismatch Negativity) /Author (Tuomo Sipola) /Subject (Advanced EEG signal processing methods) /Keywords (electroencephalography, EEG, event-related potential, ERP, mismatch negativity, MMN, Hilbert-Huang transform, HHT, empirical mode decomposition, EMD)}

%%% Actual content begins here

\title{Applying Hilbert-Huang Transform \\ to Mismatch Negativity}
\translatedtitle{Hilbert-Huang-muunnoksen soveltaminen aivos\"{a}hk\"{o}signaaliin}

\linja{Mobile Systems}

\setauthor{Tuomo}{Sipola}
\yhteystiedot{tuomo.s.sipola@jyu.fi}
%\setdate{30}{8}{2008}

\abstract{EEG signals can be analyzed with modern mathematical methods
  in order to separate the most meaningful components from the
  rest. Hilbert-Huang transform is a new method to construct a sharp
  and clean time-frequency spectrum of a nonlinear and nonstationary
  signal. It consists of empirical mode decomposition (EMD), which
  decomposes the signal to intrinsic mode functions (IMF), and Hilbert
  transform, which is used to obtain the spectrum. This method was
  used on EEG data recorded during an oddball paradigm test. The
  subjects consisted of children divided into three groups:
  attention-deficit hyperactivity disorder (ADHD), reading-disabled
  (RD) and control group. Hilbert-Huang transform revealed differences
  between the groups that could not have been obtained using more
  conventional analysis methods. }
\tiivistelma{Aivosähkökäyriä, bzzzz!
}

\avainsanat{aivos\"{a}hk\"{o}, her\"{a}tepotentiaali, Hilbert-Huang-muunnos, empiirinen aaltomuotohajotelma, aika-taajuusanalyysi}
\keywords{electroencephalography, EEG, event-related potential, ERP, mismatch negativity, MMN, Hilbert-Huang transform, HHT, empirical mode decomposition, EMD, time-frequency analysis}

\begin{document}

\def\termlistname{Abbreviations}
\termlist

\begin{abbrlist}{ANOVA}
\item[OMG] Oh my god
\item[LOL] Lulz
\end{abbrlist}

\mainmatter

\chapter{Introduction}


The structure of this thesis is following. In chapter
\ref{CHAPTER:EEG} the basics of EEG will be explained. Event-related
potentials (ERP) and their components are introduced in chapter
\ref{CHAPTER:ERP} along with a special case of ERP, mismatch
negativity (MMN). After the basic biomedical concepts the focus will
shift to data analysis methods. Chapter \ref{CHAPTER:LINEAR} takes a
brief look at the more traditional linear time-frequency methods used
in EEG analysis. Empirical mode decomposition and intrinsic mode
functions are discussed in chapter \ref{CHAPTER:EMD}. Theoretical
background behind Hilbert transform and Hilbert spectrum are
introduced in chapter \ref{CHAPTER:HT_HS}.
Jeejee.


\chapter{Electroencephalography}
\label{CHAPTER:EEG}

During the 1990s EEG was used increasingly with alternative neuroimagining techniques. Intensive care units were deployed to monitor EEG in real time in hospital environments. Complementary methods were invented but EEG has superior time resolution and remains in use \cite{LUCK2005, SWARTZ1998b}. 

\section{Brain rhythms}

Some recurrent parts of EEG have a clear origin and frequency. These repeating rhythms can be rather easily observed and they are associated with different states of alertness. For these reasons the rhythms were discovered early in EEG research and have been used as diagnostic tools and ways to characterize an EEG waveform. The most common rhythms are historically named beta ($13-30Hz$), alpha ($8-13Hz$), theta ($4-8Hz$) and delta ($0.5-4Hz$) \cite[pp. 10--12]{SANEI2007}. 

%% \begin{figure}[htp]
%% \centering
%% \includegraphics[width=150pt]{pics/electrode_positions.pdf}
%% \caption[Conventional electrode positioning]{Conventional 10--20 electrode positions. Figure adapted from Sanei and Chambers \cite[p. 16]{SANEI2007}.}
%% \label{ELECTRODE_POSITIONS}
%% \end{figure}

From the probability point of view a stochastic process $X=X(t)$ is stationary if vector $\mathbf{X}(t)=(X(t_1), X(t_2), \dots, X(t_n))$ has the same cumulative distribution as $\mathbf{X}(t+s)=(X(t_1+s),X(t_2+s),\dots,X(t_n+s))$, that is, 

\begin{equation}
F_{\mathbf{X}(t+s)}(\textbf{x}) = F_{\mathbf{X}(t)}(\textbf{x})
\label{CUM_DIST}
\end{equation}

The algorithm consists of the following steps: 

\begin{enumerate_no_space}
\item find the extrema of $x(t)$ \label{FIRST}
\item create envelope $E_u(t)$ by interpolating between the maxima ($E_l(t)$ for minima)\label{ENV}
\item mean envelope $m(t) = (E_u(t)+E_l(t))/2$
\item extract the details $c(t) = x(t) - m(t)$ \label{DETAIL}
\item go to \ref{ENV} until $c(t)$ is considered IMF \label{SIFT}
\item iterate from the start with the residue signal $r(t) = x(t) - c(t)$.
\end{enumerate_no_space}

Jee.

\begin{table}
\centering
\begin{tabular}{c c c c c c c}

\textbf{Group} & \textbf{Number} & \textbf{Boys} & \textbf{Girls} & \textbf{Min age} & \textbf{Mean age} & \textbf{Max age} \\ \hline
RD      & 16 & 11 & 5  & 8y8m & 12y2m  & 14y2m \\
ADHD    & 16 & 15 & 1  & 9y2m & 11y0m  & 13y5m \\
Control & 66 & 41 & 25 & 8y2m & 11y11m & 16y9m \\

\end{tabular}
\caption{Subjects were divided into three groups}
\label{SUBJECTS}
\end{table}


\begin{thebibliography}{99}

\bibitem{IDS}
S. Mukkamala, A.H. Sung, \textit{A Comparative Study of Techniques for Intrusion Detection} kirjassa Tools with Artificial Intelligence, 
15th IEEE International Conference, 2003.

\bibitem{WEBS}
B. Shweta, \textit{Web Security Basics}, Course Technology, 2002.

\bibitem{DDOS}
D. Dittrich, S. Dietrich, J. Mirkovic, P. Reiher, \textit{Internet Denial of Service: Attack and Defence Mechanisms}, Prentice Hall PTR, 2004.

\bibitem{DDOSb}
S. Ranjan, R. Swaminathan, M. Uysal, E. Knightly, \textit{DDoS-Resilient Scheduling to Counter Application Layer Attacks under Imperfect Detection},
Infocom 2006, 25th IEEE International Conference on Computer Communications, 2006.

\bibitem{FBI}
FBI, \textit{Wanted by the FBI - Saad Echouafni}, <URL: \texttt{http://www.fbi.gov/wanted/fugitives/cyber/echouafni\_s.htm}>, viitattu 27.10.2009.

\bibitem{CERT}
CERT, \textit{Denial of Servce}, saatavilla WWW-muodossa <URL: \texttt{http://www.cert.org/tech\_tips/denial\_of\_service.html}>, viitattu 20.10.2009.

\bibitem{STACK}
R. Bandes, B. Franklin, M. Gregg, G. Mays, C. Ries, S. Watkins, \textit{Hack the Stack}, Syngress Publishing Inc, 2006. 

\bibitem{IDSb}
R. Lippmann, D. Stetson, S. Webster, \textit{The Effect of Identifying Vulnerabilities and Patching Software on the Utility of Network Intrusion Detection},
kirjassa Recent Advances in Intrusion Detection, 2002.

\bibitem{TCP}
K. Kaario, \textit{TCP/IP-verkot}, Docendon, 2001.

\bibitem{CVE}
CVE, <URL: \texttt{http://www.cve.mitre.org/cve}>, viitattu 29.10.2009.

\bibitem{SYM}
Symantec, \textit{Symantec Internet Security Threat report. Trends for July-December 2007}, saatavilla WWW-muodossa
<URL: \texttt{http://www.symantec.com/business/index.jsp}>, viitattu 1.12.2009.

\bibitem{SYM2}
Symantec, \textit{Symantec Global Internet Security Threat Report. Trends for 2008}, saatavilla WWW-muodossa <URL: \texttt{http://www.symantec.com/business/index.jsp}>,
viitattu 1.12.2009.


\bibitem{U2R}
Z. Bankovic, S. Bojanic, O. Nieto-Taladriz, A. Badii, \textit{Increasing Detection Rate of User-to-Root Attacks Using Genetic Algorithms}, 
Emerging Security Information, Systems, and Technologies, 2007.

\bibitem{SQL SS}
C. Andrews, D. Litchfield, \textit{SQL Server Security}, McGraw-Hill Osborne, 2003.

\bibitem{WEB2}
R. Cannings, H. Dwidedi, Z. Lackey, \textit{Hacking Exposed Web 2.0: Web 2.0 Security Secrets and Solutions}, McGraw-Hill Companies, 2008.

\bibitem{WEB2b}
S. Shreeraj, \textit{Web 2.0 Security: Defending Ajax, RIA and SOA}, Course Technology, 2007.

\bibitem{WEB2c}
G. Lawton, \textit{Web 2.0 Creates Security Challenges}, Computer-IEEE Computer Society, saatavilla WWW-muodossa <URL: \texttt{http://www.computer.org}>,
viitattu 5.11.2009.

\bibitem{SOA}
IBM Redbooks, \textit{Patters: Service-Oriented Architecture and Web Services}, IBM, 2004.

\bibitem{Hacking}
S. McClure, J. Scambray, G. Kurtz, \textit{Hacking Exposed (5th Edition)}, McGraw-Hill Osborne, 2005.

\bibitem{XSS}
XXSed, <URL:\texttt{http://http://www.xssed.com/archive/special=1}>, viitattu 2.12.2009.

\end{thebibliography}


\appendix

%\chapter{Published article}
%
%\includepdf[pages=-]{published_article.pdf}

\end{document}

