\chapter{Electroencephalography}
\label{CHAPTER:EEG}

During the 1990s EEG was used increasingly with alternative neuroimagining techniques. Intensive care units were deployed to monitor EEG in real time in hospital environments. Complementary methods were invented but EEG has superior time resolution and remains in use \cite{LUCK2005, SWARTZ1998b}. 

\section{Brain rhythms}

Some recurrent parts of EEG have a clear origin and frequency. These repeating rhythms can be rather easily observed and they are associated with different states of alertness. For these reasons the rhythms were discovered early in EEG research and have been used as diagnostic tools and ways to characterize an EEG waveform. The most common rhythms are historically named beta ($13-30Hz$), alpha ($8-13Hz$), theta ($4-8Hz$) and delta ($0.5-4Hz$) \cite[pp. 10--12]{SANEI2007}. 

%% \begin{figure}[htp]
%% \centering
%% \includegraphics[width=150pt]{pics/electrode_positions.pdf}
%% \caption[Conventional electrode positioning]{Conventional 10--20 electrode positions. Figure adapted from Sanei and Chambers \cite[p. 16]{SANEI2007}.}
%% \label{ELECTRODE_POSITIONS}
%% \end{figure}

From the probability point of view a stochastic process $X=X(t)$ is stationary if vector $\mathbf{X}(t)=(X(t_1), X(t_2), \dots, X(t_n))$ has the same cumulative distribution as $\mathbf{X}(t+s)=(X(t_1+s),X(t_2+s),\dots,X(t_n+s))$, that is, 

\begin{equation}
F_{\mathbf{X}(t+s)}(\textbf{x}) = F_{\mathbf{X}(t)}(\textbf{x})
\label{CUM_DIST}
\end{equation}

The algorithm consists of the following steps: 

\begin{enumerate_no_space}
\item find the extrema of $x(t)$ \label{FIRST}
\item create envelope $E_u(t)$ by interpolating between the maxima ($E_l(t)$ for minima)\label{ENV}
\item mean envelope $m(t) = (E_u(t)+E_l(t))/2$
\item extract the details $c(t) = x(t) - m(t)$ \label{DETAIL}
\item go to \ref{ENV} until $c(t)$ is considered IMF \label{SIFT}
\item iterate from the start with the residue signal $r(t) = x(t) - c(t)$.
\end{enumerate_no_space}

Jee.

\begin{table}
\centering
\begin{tabular}{c c c c c c c}

\textbf{Group} & \textbf{Number} & \textbf{Boys} & \textbf{Girls} & \textbf{Min age} & \textbf{Mean age} & \textbf{Max age} \\ \hline
RD      & 16 & 11 & 5  & 8y8m & 12y2m  & 14y2m \\
ADHD    & 16 & 15 & 1  & 9y2m & 11y0m  & 13y5m \\
Control & 66 & 41 & 25 & 8y2m & 11y11m & 16y9m \\

\end{tabular}
\caption{Subjects were divided into three groups}
\label{SUBJECTS}
\end{table}
