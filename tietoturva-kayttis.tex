% -*- mode: LaTeX; coding: utf-8; -*-
%
% Luento kurssille TIEA222 Kokkolassa 21.8.2010.
% Joel Lehtonen
%
% Based on Timo Männikkö's template. 
%

\documentclass[a4paper,12pt]{article}

\usepackage[utf8]{inputenc}
%\usepackage[latin1]{inputenc}
\usepackage[T1]{fontenc}
\usepackage[finnish]{babel}

\usepackage{palatino} % Better font
\usepackage[intlimits]{amsmath} % For better math mode, namely integrals
\usepackage{amssymb} % Math symbols

\usepackage[pdftex]{graphicx} % Pictures

\usepackage{hyperref}
\usepackage{cite}

% Optional 1,5 point line spacing
%\usepackage{setspace}
%\onehalfspace

% Hakkeroidaan babel-paketin suomenkielisiä käännöksiä:
% Vaihdetaan lähdeluettelon otsikoksi Lähteet
\addto\captionsfinnish{\def\refname{Lähteet}}
% Hakkerointi päättyy

\title{Käyttöjärjestelmiin tunkeutumisen tapoja}
\author{Joel Lehtonen\\ \texttt{joel.lehtonen@iki.fi} \\\\Kristian Siljander\\ \texttt{kristian.siljander@jyu.fi}}
\date{23.8.2010}

\newcommand{\license}{
\bigskip%
\noindent\begin{tabular}{ l p{10cm} } %
  \hline%
  \noalign{\bigskip} %
  \includegraphics[height=1.0truecm]{cc-by-sa} %
  & \vspace{-0.95truecm}\small Tämän teoksen käyttöoikeutta koskee Creative Commons
  Attribution-ShareAlike 3.0 Unported -lisenssi.\normalsize \\ %
\end{tabular}%
}

\begin{document}

\maketitle

\begin{abstract}
\noindent Tämä on Tietoturva (TIEA222) -kurssin oheismateriaali. Tässä esitellään hyvin lyhyesti Remote-to-Local- ja User-to-Root-hyökkäykset.
\end{abstract}

\license
\pagebreak


\section{Remote-to-Local}

Remote-to-Local (lyh. \textit{R2L}) on hyökkäystyyppi, jossa hyökkääjä pyrkii
saamaan koneelle laajemmat oikeudet, kuin hänellä muuten
olisi. Tämä tapahtuu useimmiten käyttäen hyväksi järjestelmässä olevia
heikkouksia, joiden avulla hyökkääjä pääsee verkon yli murtautumaan
koneelle \cite{IDS}. Pahimmassa tapauksessa hyökkääjä saa hankittua koneelle
pääkäyttäjän oikeudet, jolloin koneen ja verkon resurssit ovat täysin
hyökkääjän käytettävissä.

Onnistuneet R2L-hyökkäykset ovat verkon ylläpitäjien kannalta pahimpia
mahdollisia, sillä niiden mahdollistamat tuhot ja aiheuttamat kustannukset ovat
huomattavasti suurempia verrattuna muihin hyökkäystyyppeihin \cite{IDSb}. Onnistunut R2L-hyökkäys saattaa
myös muut lähiverkon koneet vaaraan, sillä usein hyökkääjä pyrkii asentamaan
koneisiin ohjelmistoja, joiden avulla hyökkääjä pystyy ottamaan koneen haltuun
käyttäjän huomaamatta. Tällä tavoin osa aikaisemmin mainituista
zombverkostoista saa alkunsa.

Käytetyimmät Web-palvelinohjelmistot, Apache ja IIS, vastaavat noin 85~\%
kaikista käytetyistä palvelinsovelluksista. Näiden kahden lisäksi
BIND-\-ohjelmistoa käytetään valtaosassa nimipalvelimista. Ohjelmistojen yleisyydestä
johtuen yli puolet R2L-hyökkäyksen mahdollistavista heikkouksista onkin
löydetty näille alustoille \cite{IDS}. Useat haavoittuvaisuudet johtuvat
ohjelmointivirheistä, joiden johdosta hyökkääjä pystyy aiheuttamaan
sovellukseen muistin ylivuodon (engl. \textit{Buffer Overflow}). Tämä usein kaataa
sovelluksen tai saattaa sen sellaiseen tilaan, jossa hyökkääjä pystyy ajamaan
omia komentoja koneella. Kattava listaus löydetyistä haavoittuvuuksista ja näiden
korjauksista löytyy osoitteesta \url{www.cve.mitre.org/cve} \cite{CVE}.

Tunnetuilta R2L-hyökkäyksiltä suojautuminen on hyvin yksinkertaista, sillä
nykyinen trendi on, että haavoittuvuuden löytänyt taho ilmoittaa siitä
yleensä ensin
sovelluksen kehittäjille ennen julkista ilmoitusta. Siksi korjaus
haavoittuvuuteen on usein olemassa ennen kuin sitä on mahdollista hyödyntää \cite{IDSb}.
Vastuu jääkin verkon ylläpitäjälle, että käytetyt sovellukset ja 
suojausjärjestelmät pidetään ajan tasalla, sillä suurin osa onnistuneista hyökkäyksistä 
johtuu siitä, että tunnettuja tietoturva-aukkoja ei ole korjattu.

\section{User-to-Root}

User-to-Root (lyh. \textit{U2R}) hyökkäyksessä murtautuja pyrkii hankkimaan
koneelle pääkäyttäjän oikeudet. Tämä tapahtuu käyttäen järjestelmässä
olevia haavoittuvaisuuksia, joita ei ole paikattu. Useimmiten
hyökkäykset pohjautuvat koodausvirheisiin, jotka mahdollistavat
ylivuodon aiheuttamisen sekä odottamattomien syötteiden antamisen
\cite{IDS}. Käyttäjästä pääkäyttäjäksi hyökkäys eroaa R2L hyökkäyksestä
siten, että hyökkääjällä on jo valmiiksi pääsy koneelle tavallisen
käyttäjän oikeuksilla.

U2R-hyökkäykset ovat L2R-hyökkäysten ohella vaikeimpia torjua, jos kohteena
oleva järjestelmä on altis U2R-hyökkäyksille. Vaikea torjuttavuus johtuu siitä, että
usein hyökkäyksestä aiheutuva liikenne muistuttaa hyvin paljon normaalia
liikennettä, jonka vuoksi itsestään oppivat puolustusjärjestelmät eivät pysty
riittävän tarkasti erottamaan haitallista liikennettä normaalista \cite{U2R}.
Kehittyneimmilläkin järjestelmillä näiden hyökkäysten tunnistaminen on todella
heikkoa. Tämän on osoittanut useat eri tutkimushankkeet, jotka ovat käyttäneet
järjestelmiensä testaamiseen DARPAn simuloimaa verkkoliikennettä,
jossa olevat tietomurrot ja muu poikkeava toiminta on merkitty.
Parhaimmillaankin tunnistaminen on jäänyt 20 prosentin
paikkeille.

\bibliographystyle{ieeetr}
\bibliography{lahteet}

\end{document}
