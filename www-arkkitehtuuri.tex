% -*- mode: LaTeX; coding: utf-8; -*-

\chapter{Web-palvelut}

Web-palveluilla tarkoitetaan järjestelmiä, jotka kommunikoivat
keskenään käyttäen vakiintuneita Web-tekniikoita. Web-palvelimet eivät
ole riippuvaisia mistään tietystä laitteisto- tai
käyttöjärjestelmäarkkitehtuurista. Web-palveluiden käyttämät
teknologiat eivät sinänsä ole erityisen vallankumouksellisia, mutta
tiedonvaihdon helpottuminen standardien protokollien ja Web:n
hajautetun rakenteen ansiosta on tehnyt Web-palveluista
mielenkiintoisia niin kehittäjien kuin käyttäjienkin
näkökulmasta\cite{javaweb}.

IBM määrittelee WWW-palvelut seuraavasti: ``Web-palvelut ovat
itsenäisiä ja modulaarisia sovelluksia, jotka voidaan julkistaa,
määritellä, paikallistaa ja suorittaa verkon ylitse, yleeensä WWW:n
välityksellä.''\cite[s.4]{websecurity}

Tietoturvan kannalta WWW-palvelut ovat haastavia, koska niitä
käytetään Internetin välityksellä eivätkä ne ole rajoittuneita
esimerkiksi tietyn organisaation lähiverkkoon. Myös osia
WWW-palveluista sijaitsee usein fyysisesti eri paikoissa ja ne
kommunikoivat keskenään Internetin välityksellä.

Tässa luvussa käsitellään komponentteja, joista tämänhetkiset
Web-palvelut koostuvat. Erityistä huomiota on kiinnitetty seikkoihin,
jotka vaikuttavat palvelun tietoturvaan.

\section{Palvelinohjelmistot}

Web-palveluiden toteuttajan näkökulmasta lähin komponentti on
palvelinohjelmisto. Se käsittelee HTTP-kyselyt ja lähettää vastauksina
käyttäjän selaimessa näytettäviä ja prosessoitavia osia, kuten
hypertekstiä, kuvia ja komentosarjoja.

Web-palvelinohjelmisto ei välttämättä kommunikoi suoraan palvelun
käyttäjän tietokoneen kanssa vaan välissä käytetään usein välimuistipalvelua,
joka vähentää palvelimelle kohdistuvaa rasistusta säilyttämällä
harvoin muuttuvaa sisältöä välimuistissa ja välittämällä tätä suoraan
käyttäjille. Muuttuva tai välimuistille uusi sisältö pyydetään
palvelinohjelmistolta.

Internet-tietoturvapalveluita tarjoavan Netcraftin raportista
huhtikuulta 2010~\cite{netcraft} käy ilmi, että kaksi suurinta
Web-palvelinalustaa ovat \textit{Apache} (53.93~\% suosituimmista
sivuista), \textit{Microsoft} (24.97~\%). Näiden jälkeen tulevat
\textit{Google} ja \textit{nginx} kumpikin 6~\% osuudella.

Tuloksia voi vääristää jonkin verran se, että joissakin tapauksissa
edustalla suoritetaan välimuistipalveluna toista palvelinohjelmistoa,
esimerkiksi Apachea tai nginx:ä, joka välittää kyselyt eteenpäin
varsinaiselle Web-palvelinohjelmistolle. Tämä selittää esimerkiksi
sen, miksi Zope puuttuu kokonaan suosituimpien palvelinten
joukosta. Esimerkiksi Jyväskylän yliopiston Web-palvelimena näkyy
tilastoinnissa Apache, vaikka Web-palvelu on toteutettu
Zope-palvelinohjelmiston päällä suoritettavalla Plonella. Samoin on
käynyt Tomcat-palvelimella ajettavalle Korppi-o\-pin\-to\-tie\-to\-jär\-jes\-tel\-mäl\-le.

\subsection{Apache}

Suosituimpana palvelinalustana Apache on luonnollisesti suosittu kohde
myös tietomurtojen yrityksille. Apachesta itsestään on kuitenkin
löydetty verrattain vähän
tietoturva-aukkoja. Suurin osa tietomurroista
keskittyykin murtamaan varsinaista Web-palvelua palvelinalustan
sijaan. (tämä on oletus, FIXME etsi lähde tai poista)

Apache pystyy suorittamaan komentosarjoja perinteisen CGI-rajapinnan
lisäksi myös palvelinlaajennosten avulla, jolloin saavutetaan
tiiviimpi integraatio ja enemmän suorituskykyä verrattuna
CGI-rajapintaan~\cite{cginopeus}.

Apachea voidaan suorittaa lukuisissa eri käyttöjärjestelmissä,
mukaanlukien Linux ja Windows. Suosittuja Apache-alustalla
käytettäviä ohjelmointikieliä ovat PHP ja Python. Tunnettuja tässä
ympäristössä käytettäviä Web-palveluita ovat mm. Wikipediassa
käytettävä MediaWiki ja lukuisilla selainkäyttöisillä
keskustelupalstoilla käytettävä phpBB.

TODO LAMP (Linux + Apache + MySQL + PHP)

\todo{Nämä tulevat viikolla 22.}

\subsection{Zope}

+Plone. TODO

\subsection{Tomcat}

TODO

\subsection{IIS}

TODO.

\section{Hajautetut palvelut}

TODO Pyynnön ohjaaminen usealle palvelimelle (reverse proxy), Hyödyt verrattuna keskitettyyn

\section{Välimuistipalvelut}

TODO 
- "talon sisällä" olevat välimuistit
 - Akamai ym. "ulkoiset" välimuistit (ei luovuta esim. IP-osoitteita)
 - tiedon keräämisen kannalta IDS:lle (minkälaiset logitukset?)

