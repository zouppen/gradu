% -*- mode: LaTeX; coding: utf-8; -*-

\chapter{WWW-palvelun arkkitehtuuri}

WWW-palveluilla tarkoitetaan järjestelmiä, jotka kommunikoivat
keskenään käyttäen vakiintuneita Web-tekniikoita. Web-palvelimet eivät
ole riippuvaisia mistään tietystä laitteisto- tai
käyttöjärjestelmäarkkitehtuurista. WWW-palveluiden käyttämät
teknologiat eivät sinänsä ole erityisen vallankumouksellisia, mutta
tiedonvaihdon helpottuminen standardien protokollien ja WWW:n
hajautetun rakenteen ansiosta on tehnyt WWW-palveluista
mielenkiintoisia niin kehittäjien kuin käyttäjienkin
näkökulmasta\cite{javaweb}.

IBM määrittelee WWW-palvelut seuraavasti: ``Web-palvelut ovat
itsenäisiä ja modulaarisia sovelluksia, jotka voidaan julkistaa,
määritellä, paikallistaa ja suorittaa verkon ylitse, yleeensä WWW:n
välityksellä.''\cite[s.4]{websecurity}

Tietoturvan kannalta WWW-palvelut ovat haastavia, koska niitä
käytetään Internetin välityksellä eivätkä ne ole rajoittuneita
esimerkiksi tietyn organisaation lähiverkkoon. Myös osia
WWW-palveluista sijaitsee usein fyysisesti eri paikoissa ja ne
kommunikoivat keskenään Internetin välityksellä.

TODO.
