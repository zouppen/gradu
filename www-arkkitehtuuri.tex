% -*- mode: LaTeX; coding: utf-8; -*-

\chapter{Web-palvelut}

Web-palveluilla tarkoitetaan järjestelmiä, jotka kommunikoivat
keskenään käyttäen vakiintuneita Web-tekniikoita. Web-palvelimet eivät
ole riippuvaisia mistään tietystä laitteisto- tai
käyttöjärjestelmäarkkitehtuurista. Web-palveluiden käyttämät
teknologiat eivät sinänsä ole erityisen vallankumouksellisia, mutta
tiedonvaihdon helpottuminen standardien protokollien ja Web:n
hajautetun rakenteen ansiosta on tehnyt Web-palveluista
mielenkiintoisia niin kehittäjien kuin käyttäjienkin
näkökulmasta\cite{javaweb}.

IBM määrittelee WWW-palvelut seuraavasti: ``Web-palvelut ovat
itsenäisiä ja modulaarisia sovelluksia, jotka voidaan julkistaa,
määritellä, paikallistaa ja suorittaa verkon ylitse, yleeensä WWW:n
välityksellä.''\cite[s.4]{websecurity}

Tietoturvan kannalta WWW-palvelut ovat haastavia, koska niitä
käytetään Internetin välityksellä eivätkä ne ole rajoittuneita
esimerkiksi tietyn organisaation lähiverkkoon. Myös osia
WWW-palveluista sijaitsee usein fyysisesti eri paikoissa ja ne
kommunikoivat keskenään Internetin välityksellä.

Tässa luvussa käsitellään komponentteja, joista tämänhetkiset
Web-palvelut koostuvat. Erityistä huomiota on kiinnitetty seikkoihin,
jotka vaikuttavat palvelun tietoturvaan.

\section{Palvelinohjelmistot}

Web-palveluiden toteuttajan näkökulmasta lähin komponentti on
palvelinohjelmisto. Se käsittelee HTTP-kyselyt ja lähettää vastauksina
käyttäjän selaimessa näytettäviä ja prosessoitavia osia, kuten
hypertekstiä, kuvia ja komentosarjoja.

Web-palvelinohjelmisto ei välttämättä kommunikoi suoraan palvelun
käyttäjän tietokoneen kanssa vaan välissä käytetään usein välimuistipalvelua,
joka vähentää palvelimelle kohdistuvaa rasistusta säilyttämällä
harvoin muuttuvaa sisältöä välimuistissa ja välittämällä tätä suoraan
käyttäjille. Muuttuva tai välimuistille uusi sisältö pyydetään
palvelinohjelmistolta.

Internet-tietoturvapalveluita tarjoavan Netcraftin raportista
huhtikuulta 2010~\cite{netcraft} käy ilmi, että kaksi suurinta
Web-palvelinalustaa ovat \textit{Apache} (53.93~\% suosituimmista
sivuista), \textit{Microsoft} (24.97~\%). Näiden jälkeen tulevat
\textit{Google} ja \textit{nginx} kumpikin 6~\% osuudella.

Tuloksia voi vääristää jonkin verran se, että joissakin tapauksissa
edustalla suoritetaan välimuistipalveluna toista palvelinohjelmistoa,
esimerkiksi Apachea tai nginx:ä, joka välittää kyselyt eteenpäin
varsinaiselle Web-palvelinohjelmistolle. Tämä selittää esimerkiksi
sen, miksi Zope puuttuu kokonaan suosituimpien palvelinten
joukosta. Esimerkiksi Jyväskylän yliopiston Web-palvelimena näkyy
tilastoinnissa Apache, vaikka Web-palvelu on toteutettu
Zope-palvelinohjelmiston päällä suoritettavalla Plonella. Samoin on
käynyt Tomcat-palvelimella ajettavalle Korppi-o\-pin\-to\-tie\-to\-jär\-jes\-tel\-mäl\-le.

\subsection{Apache}

Suosituimpana palvelinalustana Apache on luonnollisesti suosittu kohde
myös tietomurtojen yrityksille. Apachesta itsestään on kuitenkin
löydetty verrattain vähän
tietoturva-aukkoja. Suurin osa tietomurroista
keskittyykin murtamaan varsinaista Web-palvelua palvelinalustan
sijaan. (tämä on oletus, FIXME etsi lähde tai poista)

Apache pystyy suorittamaan komentosarjoja perinteisen CGI-rajapinnan
lisäksi myös palvelinlaajennosten avulla, jolloin saavutetaan
tiiviimpi integraatio ja enemmän suorituskykyä verrattuna
CGI-rajapintaan~\cite{cginopeus}.

Apachea voidaan suorittaa lukuisissa eri käyttöjärjestelmissä,
mukaanlukien Linux ja Windows. Suosittuja Apache-alustalla
käytettäviä ohjelmointikieliä ovat PHP ja Python. Tunnettuja tässä
ympäristössä käytettäviä Web-palveluita ovat mm. Wikipediassa
käytettävä MediaWiki ja lukuisilla selainkäyttöisillä
keskustelupalstoilla käytettävä phpBB.

TODO LAMP (Linux + Apache + MySQL + PHP)

\todo{Nämä tulevat viikolla 22.}

\subsection{IIS}

Apachen jälkeen toiseksi suurinta markkinaosuutta web-palvelinalustoista pitää Microsoftin kehittämä Internet Information Services (lyh. IIS). Ensimmäisen versio sovelluksesta
julkaistiin NT-käyttöjärjestelmälle vuona 1996, ja vuonna 2008 julkaistiin IIS 7 Windows Server 2008 palvelinjakelun yhteydessä. Uusin virallinen versio sovelluksesta on tällä hetkellä   
IIS 7.5, ja siitä on ladattavissa 180-päivän ilmainen kokeiluversio Windows Server 2008 alustalle Microsoftin sivuilta \cite{IIS}.

Aikaisemmista IIS-versioista on löydetty paljon eri tasoisia tietoturvariskejä, joista tunnetuin on Code Red Wormiksi nimetty hyökkäys, joka saastutti yli kolmesataatuhatta web-sivustoa.
Hyökkäys pohjautui puskuriylivuotoon, johon Microsoft oli julkaissut päivityksen kuukausi takaperin, mutta jota osa palveluita ylläpitävistä tahoista ei ollut asentanut. Tämän hyökkäyksen
johdosta yleinen käsitys IIS-sovellusten tietoturvasta on vähintäänkin kyseenalainen. Uusimmista versioista ei ole kuitenkaan löydetty vastaavanlaisia tietoturvariskejä, ja Secunian
ylläpitämän listan mukaan uusimmasta versiosta ei löydy yhtään paikkaamatonta tietoturvariskiä \cite{Secunia}.

Suurin muutos uusimmissa versioissa verrattaessa vanhoihin on siirtyminen modulaariseen arkkitehtuuriin. Tämä mahdollistaa uusien komponenttien nopean lisäämisen ja poistamisen, jonka lisäksi  
monipuolinen API-rajapinta tarjoaa mahdollisuuden tehdä yksilöllisiä komponentteja omiin tarpeisiin. Modulaarinen rakenne parantaa myös sovelluksen tietoturvaa pienentämällä asennettavien 
osien määrää ainoastaan niihin, joita ylläpitäjä tarvitsee. Tietoturvaan ja skaalautuvuuteen on muutenkin kiinnitetty enemmän huomiota alusta asti, joiden lisäksi monet Apachen ominaisuuksista
kuten URL-osoitteiden uudelleenkirjoitus on otettu käyttöön \cite{IIS}. IIS:n suurimpana haittapuolena nykyisin sen onkin sidonnaisuus Windows-maailmaan, sillä muille alustoille sitä ei ole 
saatavilla. Tästä huolimatta IIS on nykyisin varteenotettava vaihtoehto Apachelle tietynlaisessa ympäristössä. 

\subsection{Zope}

Vaikka Apachen ja Microsoft IIS:n markkinaosuudet kattavat suurimman osan käytetyistä web-alustoista, on markkinoilla suuri määrä pienempiä ratkaisuja, joille löytyy oma käyttäjäkuntansa. Yksi 
näistä on vapaaseen lähdekoodiin pohjautuva Zope \cite{Zope}. Zope tulee sanoista ``Z Object Publishing Environment'', ja se on kirjoitettu pääosin käyttäen Pythonia. Zope on ensimmäinen
web-sovellusten suunnitteluun tarkoitettu oliopohjainen julkaisujärjestelmä, ja sen mukana tulee erillinen tietokantasovellus ja web-palvelinalusta. Zopesta on olemassa useita eri jakeluita, ja 
näistä käytetyin on Zope2. 

Vaikka Zope itsessään sisältää web-palvelimen, voidaan sitä käyttää myös rinnakkain muiden palvelimien kanssa. Tällaiseen ratkaisuun voidaan päätyä, jos halutaan esimerkiksi ylläpitää samalla
palvelimella sekä Zopella että muilla työkaluilla tuotettua sisältöä, tai jos halutaan käyttää palveluita kuten Apachen SSL. Monissa tehtävissä muut alustat toimivat myös nopeammin ja 
turvallisemmin kuin Zope. Zopen käyttö ei myöskään sulje perinteisen tietokantaratkaisun käyttöä, ja sen tukemiin tietokantoihin kuuluu muun muassa Oracle, PostgreSQL ja MySQL. Usein eri 
tietokantoja käytetään myös rinnakkain, jolloin Zopen objektit voidaan tallentaa ZODB-oliotietokantaan ja muu data perinteisiin relaatiotietokantoihin. 

Zopen yksi suurimmista vahvuuksista on sille tehdyt lisäosat, joita löytyy lukematon määrä. Näistä tunnetuin on Zopen päälle rakennettu Plone \cite{Plone}, joka on pitkälle kehitetty 
sisällönhallintajärjestelmä. Sitä voidaan käyttää kaikenlaisen web-sisällön kuten blogien ja sisäisten ja ulkoisten web-sivustojen hallitsemiseen. Se on helppo asentaa ja ottaa käyttöön, 
jonka lisäksi se on käännetty yli 40 kielelle. Helppokäyttöisyys, joustavuus ja laajennettavuus ovat niitä tekijöitä, jotka ovat tehneet Plonesta suositun. Plonelle löytyvä ohjeistus
on myös hyvin kattavaa, ja se on yksi tuetuimmista avoimen lähdekoodin projekteista. Ilmaista web-sovellusalustaa etsivälle Zope ja Plone tarjoavatkin parasta mahdollista.

\section{Hajautetut palvelut}

TODO Pyynnön ohjaaminen usealle palvelimelle (reverse proxy), Hyödyt verrattuna keskitettyyn

\section{Välimuistipalvelut}

TODO 
- "talon sisällä" olevat välimuistit
 - Akamai ym. "ulkoiset" välimuistit (ei luovuta esim. IP-osoitteita)
 - tiedon keräämisen kannalta IDS:lle (minkälaiset logitukset?)

