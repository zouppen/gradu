% -*- mode: LaTeX; coding: utf-8; -*-

\chapter{Johdanto}

Viime vuosina web-sovellukset ovat kasvattaneet suuresti suosiota, ja nykyisin yhä useampi palvelu, jossa tietoturvan ja saatavuuden takaaminen on elintärkeää, on siirtynyt osittain tai kokonaan 
verkon puolelle. Tämä on vaikuttanut suuresti siihen, mihin tietoturvahyökkäykset nykyisin kohdistuvat, ja kuinka niitä pyritään toteuttamaan. Hyökkäysten muuttuessa yhä hienostuneimmiksi, 
eivät perinteiset tietoturvamenetelmät enää riitä suojaamaan loppukäyttäjiä tai palvelun ylläpitäjiä. Tästä syystä erilaisten tietoturvaratkaisuiden ympärillä käy kova kuhina, ja aihepiiri 
on herättänyt suurta kiinnostusta tutkijoiden keskuudessa. Erilaisia ratkaisuja, joissa on pyritty selvittämään tietoturvahyökkäyksiä ja näiden mukana tulevia haasteita, on lukematon määrä. 
Ongelma on löytää näistä ne menetelmät, jotka oikeasti toimivat riittävällä tarkkuudella ja nopeudella.  

Hyökkäysten tunnistaminen toimii siten, että analysoimalla yhtä tai useampaa tapahtumaa pyritään löytämään viitteitä tapahtuvista hyökkäyksistä. Tunnistamiseen pohjautuvat menetelmät jaetaan usein
kahteen eri tyyppiin: anomalioiden eli poikkeavuuksien tunnistamiseen (eng. anomaly detection) ja väärinkäytösten tunnistamiseen (eng. misuse detection). Näistä anomalioiden tunnistamiseen perustuvat
järjestelmät luovat malleja järjestelmän, käyttäjän tai verkon normaalista käyttäytymisestä, ja vertaamalla tapahtumia näin muodostettuihin kuvauksiin, voidaan poikkeavuudet tunnistaa. Väärinkäytösten 
tunnistamiseen tarkoitetut järjestelmät taas sisältävät joukon kuvauksia eli signatuureja tunnetuista hyökkäyksistä. Tuleva liikenne tarkistetaan näitä kuvauksia vastaan, jolloin kuvauksia vastaavat
hyökkäykset voidaan tunnistaa. Joissakin tapauksissa jaottelu on myös tehty sen mukaan, mistä tutkittava liikenne on kerätty. Tällöin järjestelmät on jaettu verkkoon pohjautuviksi (eng. network based)
ja asiakaspohjaisiksi (eng. host based). 

