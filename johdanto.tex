% -*- mode: LaTeX; coding: utf-8; -*-

\chapter{Johdanto}

Viime vuosina Web-sovellusten ja -palveluiden suosio ja käyttö ovat
olleet kovassa kasvussa, ja monet näistä ovat nykyisin kriittisiä osia
yhteiskuntamme toimivuuden kannalta. Internetin välityksellä
käytettävien palveluiden tietoturvan ja saatavuuden takaaminen on
tämän johdosta noussut monessa yrityksessä
elintärkeäksi. Tietojärjestelmien siirtyessä julkishallinnon ja
yritysten erillisverkoista julkiseen Internetiin on vaikuttanut omalta
osaltaan myös siihen, mihin tietoturvahyökkäykset nykyisin
kohdistuvat, ja kuinka niitä pyritään toteuttamaan.

Hyökkäysten muuttuessa yhä hienostuneimmiksi, eivät perinteiset
tietoturvamenetelmät enää riitä suojaamaan loppukäyttäjiä tai palvelun
ylläpitäjiä. Tästä syystä erilaisten tietoturvaratkaisuiden ympärillä
käy kova kuhina, ja aihepiiri on herättänyt suurta kiinnostusta
tutkijoiden keskuudessa. Erilaisia ratkaisuja, joissa on pyritty
selvittämään tietoturvahyökkäyksiä ja näiden mukana tulevia haasteita,
on lukematon määrä. Haaste on löytää ne menetelmät, jotka oikeasti
toimivat riittävällä tarkkuudella ja nopeudella.

Tietoturvahyökkäysten tunnistaminen perustuu siihen, että
analysoimalla yhtä tai useampaa tapahtumaa pyritään löytämään
viitteitä tapahtuvista hyökkäyksistä. Menetelmät jaetaan kahteen eri
tyyppiin: anomalioiden eli poikkeavuuksien tunnistamiseen
(engl. \textit{anomaly detection}) ja väärinkäytösten tunnistamiseen
(engl. \textit{misuse detection}). Näistä anomalioiden tunnistaminen
perustuu malleihin, joita luodaan järjestelmän, käyttäjän tai verkon
normaalista käyttäytymisestä. Näin luotuja malleja verrataan tuleviin
tapahtumia, jolloin poikkeavuudet voidaan tunnistaa. Väärinkäytösten
tunnistamiseen tarkoitetut järjestelmät puolestaan sisältävät joukon
kuvauksia eli signatuureja tunnetuista hyökkäyksistä. Tuleva liikenne
tarkistetaan näitä kuvauksia vastaan, jolloin kuvauksia vastaavat
hyökkäykset voidaan tunnistaa. Hyökkäysten tunnistusjärjestelmät
jaetaan joissakin tapauksissa myös sen mukaan, mistä tutkittava
liikenne on kerätty. Tällöin järjestelmät on jaettu verkkoon
pohjautuviksi (engl. \textit{network based}) ja asiakaspohjaisiksi
(engl. \textit{host based}).

Tässä tutkielmassa tietoturvahyökkäykset pyritään tunnistamaan
analysoimalla Web-palvelimien tuottamaa tapahtumalokia. Lokista
yritetään etsiä ja tunnistaa anomalioita normaalin liikenteen
joukosta.  Käytetty menetelmä pohjautuu diffuusiokuvausten käyttöön.

Tutkielman rakenne on seuraavanlainen. Luvussa 2 esitellään yleisellä
tasolla Web-palvelinalustoja, ja käydään läpi Web-palveluiden
erilaisia toteutustapoja. Luku 3 käsittelee perinteisiä
tietoturvahyökkäyksiä, ja luvussa 4 perehdytään uudenlaisiin
Web-ratkaisuihin ja tietoturvahyökkäyksiin, joiden kohteeksi
Web-palvelut nykyisin joutuvat.  Luku 5 sisältää kuvauksen
dynaamisista Web-teknologioista, jotka mahdollistavat suurimman osan
nykyisistä tietoturvahyökkäyksistä, ja luku 6 sisältää katsauksen
tutkimuskenttään liittyvään tutkimukseen.

Luvussa 7 esitellään tutkimusasetelma ja analyysissa käytetty
materiaali. Luku sisältää myös kuvauksen analyysissa käytetyistä
menetelmistä. Luvussa 8 käydään läpi lokitiedon rakennetta ja
esikäsittelijää, joka on on suunniteltu ja toteutettu tätä tutkimusta
varten. Esimerkkien avulla esitellään myös sovelluksen toimintaa.

Luvussa 9 esitellään saatuja tuloksia kuvaajien avulla, ja
analyisoidaan näitä tarkemmin sekä esitetään mahdollisia syitä
saatuihin tuloksiin. Viimeisessä luvussa sitten tehdään yhteenveto
koko työstä, ja jatkotutkimuksia ajatellen esitetään
parannusehdotuksia.
