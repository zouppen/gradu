% -*- mode: LaTeX; coding: utf-8; -*-

\chapter{Johdanto}

Viimeisen vuosikymmenen aikana Web-sovellukset ovat kasvattaneet suuresti suosiota ja monet Web-palvelut ovat nykyisin kriittisiä osia yhteiskuntamme toimivuuden kannalta. Internetin välityksellä käytetettävien palveluiden tietoturvan ja saatavuuden takaaminen onkin monissa tapauksissa elintärkeää. Tietojärjestelmien siirtyminen julkishallinnon ja yritysten erillisverkoista julkiseen Internetiin on vaikuttanut omalta osaltaan myös siihen, miten tietoturvahyökkäykset nykyisin kohdistuvat, ja kuinka niitä pyritään toteuttamaan.

Hyökkäysten muuttuessa yhä hienostuneimmiksi, 
eivät perinteiset tietoturvamenetelmät enää riitä suojaamaan loppukäyttäjiä tai palvelun ylläpitäjiä. Tästä syystä erilaisten tietoturvaratkaisuiden ympärillä käy kova kuhina, ja aihepiiri 
on herättänyt suurta kiinnostusta tutkijoiden keskuudessa. Erilaisia ratkaisuja, joissa on pyritty selvittämään tietoturvahyökkäyksiä ja näiden mukana tulevia haasteita, on lukematon määrä. 
Haasteena on löytää näistä ne menetelmät, jotka oikeasti toimivat riittävällä tarkkuudella ja nopeudella.  

Tietoturvahyökkäysten tunnistaminen toimii siten, että analysoimalla yhtä tai useampaa tapahtumaa pyritään löytämään viitteitä tapahtuvista hyökkäyksistä. Menetelmät jaetaan usein
kahteen eri tyyppiin: anomalioiden eli poikkeavuuksien tunnistamiseen (engl. \textit{anomaly detection}) ja väärinkäytösten tunnistamiseen (engl. \textit{misuse detection}). Näistä anomalioiden tunnistamiseen perustuvat malleihin, joita luodaan järjestelmän, käyttäjän tai verkon normaalista käyttäytymisestä. Kun näitä malleja verrataan tuleviin tapahtumia, voidaan poikkeavuudet tunnistaa. Väärinkäytösten 
tunnistamiseen tarkoitetut järjestelmät puolestaan sisältävät joukon kuvauksia eli signatuureja tunnetuista hyökkäyksistä. Tuleva liikenne tarkistetaan näitä kuvauksia vastaan, jolloin kuvauksia vastaavat
hyökkäykset voidaan tunnistaa. Hyökkäysten tunnistusjärjestelmät jaetaan joissakin tapauksissa myös sen mukaan, mistä tutkittava liikenne on kerätty. Tällöin järjestelmät on jaettu verkkoon pohjautuviksi (engl. \textit{network based})
ja asiakaspohjaisiksi (engl. \textit{host based}).

