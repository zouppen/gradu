% BACHELOR'S THESIS IN MATHEMATICAL INFORMATION TECHNOLOGY
% Joel Lehtonen
%
% Based on Timo Männikkö's template with some modifications made by
% Tuomo Sipola (M.Sc.).
%
% Uses latex template by Antti-Juhani Kaijanaho and Matthieu Weber.
%

\documentclass[finnish,logo,nonumbib,creativecommons,nocopyright,lof,pdftex,palatino,utf8]{gradu2}

\usepackage{palatino} % Better font
\usepackage[intlimits]{amsmath} % For better math mode, namely integrals
\usepackage{amssymb} % Math symbols

\usepackage[pdftex]{graphicx} % Pictures
\usepackage{pdfpages} % For inclusion of the article

\usepackage{hyperref}
\usepackage{cite}
\usepackage{framed}
\usepackage{verbatim} % Multi-line commenting.
\usepackage{rotating} % Vertical text used in temporary comments.
\usepackage{listings} % Source code

\tyyppi{kandidaatintutkielma} % muuten oletus on "pro gradu -tutkielma"

% Todo-tags help the examiner's work
\newcommand{\todo}[1]{\marginpar{\begin{framed}\begin{sideways}\textbf{TODO}: #1\end{sideways}\end{framed}}}

% Source code listings. Annotation can be done between [ and ] in Bash.
\lstset{frame=single,captionpos=b}
\renewcommand{\lstlistingname}{Listaus}
\lstdefinelanguage{bashshell}{language=bash,moredelim=[is][\itshape]{[[}{]]},morekeywords={mkdir,find,classifier,ls,parameter_analyzer,parameter2vector,xargs,apache2data},aboveskip=0.5cm}
\lstdefinelanguage{MyHaskell}{language=Haskell,morekeywords={ByteString,URL,UTCTime,readZipolaFile,saveLogLines}}

% Define our own abbreviations list
\newenvironment{abbrlist}[1]{
  \begin{list}{}{\settowidth{\labelwidth}{\textbf{#1}}
  \setlength{\leftmargin}{\labelwidth}
  \addtolength{\leftmargin}{\labelsep}
  \renewcommand{\makelabel}[1]{\textbf{\hfill##1}}}}%
{\end{list}}

% No spacing with this enumeration
\newenvironment{enumerate_no_space}{
  \begin{enumerate}
  \setlength{\itemsep}{1pt}
  \setlength{\parskip}{0pt}
  \setlength{\parsep}{0pt}}
{\end{enumerate}}

% Remove pagebackref in the final version
% Add "pagebackref=true" to get page numbers where each reference is used
% Link version for internet viewing
%\usepackage[pdftex,colorlinks=true]{hyperref}

% PDF information tags
\pdfinfo{/Title (Web-palveluiden tietoturva) /Author (Joel Lehtonen) /Subject (Tietoturva) /Keywords (tietoturva,web)}

%%% Actual content begins here

\title{Web-palveluiden tietoturva}
\translatedtitle{Securitu of Web Services}

\setauthor{Joel}{Lehtonen}
\yhteystiedot{joel.lehtonen@iki.fi}
%\setdate{30}{8}{2008}
\license{Tämän teoksen käyttöoikeutta koskee Creative Commons
  Attribution-ShareAlike 3.0 Unported -lisenssi.}

%-miksi kiinnostavia
%-mitä tutkittiin
%-miten tutkittiin
%-tulokset

% TODO tiivistelmä

\abstract{ In recent years the popularity and usage of Web
  applications and services have grown rapidly and nowadays many of those
  services play an important role in our information society. Because
  of this, securing Web services is essential. The complexity of
  security attacks has renderd traditional security methods almost
  useless. Therefore new methods are being developed and diffusion
  maps are one of the most promising ones. In this thesis, the
  suitability of diffusion maps in evaluation of HTTP traffic is
  researched. The results show that the used methods are capable of
  detecting anomalous HTTP traffic from large data sets. Still, for
  active intrusion detection further research is needed. }

\tiivistelma{ Viime vuosina Web-sovellusten ja -palveluiden suosio on
  ollut voimakkaassa kasvussa, ja monet palveluista ovat nykyisin
  kriittisiä osia yhteiskuntamme toimivuuden kannalta.  Tämän johdosta
  palveluiden suojaaminen on noussut tärkeään roolin. Hyökkäysten myös
  muuttuessa yhä hienostuneimmiksi, eivät perinteiset
  tietoturvamenetelmät tarjoa enää riittävää suojaa. Tämän vuoksi
  uusia menetelmiä kehitetään jatkuvasti.}

% TODO avainsanat

\avainsanat{
anomalia,
diffuusiokuvaus,
N-grammianalyysi,
tietoturva,
Web-palvelu
}
\keywords{
anomaly,
data security,
diffusion maps,
N-gram analysis,
Web service
}

\def\termlistname{Lyhenteet}

\begin{document}

% Kirjastulostusta varten fontti luettavamman kokoiseksi
%\scalefont{1.18}

% -*- mode: LaTeX; coding: utf-8; -*-

\termlist

\begin{abbrlist}{ANOV}
\item[AJAX]	Asynchronous JavaScript And XML. Joukko Web-sovelluskehityksessä käytettyjä tekniikoita.
\item[CSRF]	Cross Site Request Forgery. Web-sovelluksissa esiintyvä tietoturva-aukko, joka väärinkäyttää selaimen ja sivuston välistä luottamussuhdetta.
\item[DOM]	Document Object Model. Ohjelmointirajapinta, joka mahdollistaa HTML- ja XHTML-dokumenttien sisällön muokkauksen.
\item[HTML]	Hypertext Markup Language. Standardoitu kuvauskieli, jolla kuvataan hyperlinkkejä sisältävää tekstiä.
\item[HTTP]     Hypertext Transfer Protocol. Selainten ja Web-palvelimien käyttämä tiedonsiirtoprotokolla.
\item[ICMP]	Internet Control Message Protocol. TCP/IP-pinon kontrolliprotokolla, jolla lähetetään nopeasti viestejä koneelta toiselle.
\item[IDS]	Intrusion Detection System. Järjestelmä, joka on ohjelmoitu tunnistamaan verkkoon suuntautuvat hyökkäysyritykset.
\item[IIS]      Internet Information Services. Microsoftin kehittämä Web-palvelinalusta.
\item[IP]	Internet Protocol. Verkkokerroksen protokolla, joka huolehtii tietoliikennepakettien välittämisestä pakettikytkentäisessä Internet-verkossa.
\item[IPS]	Intrusion Prevention System. Järjestelmä, jolla pyritään ennakoivasti estämään tietomurtoyritykset.
\item[LDAP]	Lightweight Directory Access Protocol. Hakemistopalveluiden käyttöön tarkoitettu verkkoprotokolla.
\item[OSI]	Open Systems Interconnection Reference Model. Eri tiedonsiirtoprotokollista muodostuva kuvaus.
\item[SOA]	Service Oriented Architecture. Joustava arkkitehtuuriratkaisu, jolla pyritään helpottamaan eri palveluiden välistä kommunikointia.
\item[SQL]	Structured Query Language. Standardoitu kyselykieli, jolla relaatiotietokantaan voi tehdä hakuja, muutoksia ja lisäyksiä.
\item[XML]	eXtensible Markup Language. Merkintäkieli, jota käytetään sekä formaattina tiedonvälitykseen järjestelmien välillä että dokumenttien tallentamiseen.
\item[XSS]	Cross Site Scripting. Web-sovelluksissa esiintyvä tietoturva-aukko, joka mahdollistaa käyttäjän koodin syöttämisen sovellukselle.  
\item[Xpath]    XML Path Language. XML-dokumenttien osien osoittamiseen tarkoitettu kieli.	
\end{abbrlist}


\mainmatter

% -*- mode: LaTeX; coding: utf-8; -*-

\chapter{Johdanto}

Viime vuosina Web-sovellusten ja -palveluiden suosio on 
ollut voimakkaassa kasvussa, ja monet palveluista ovat nykyisin kriittisiä osia
yhteiskuntamme toimivuuden kannalta. Internetin välityksellä
käytettävien palveluiden tietoturvan ja saatavuuden takaaminen on
tämän johdosta noussut monessa yrityksessä tärkeään rooliin.
Tietojärjestelmien siirtyminen julkishallinnon ja
yritysten erillisverkoista julkiseen Internetiin on vaikuttanut omalta
osaltaan myös siihen, mihin tietoturvahyökkäykset nykyisin
kohdistuvat ja kuinka ne pyritään toteuttamaan.

Hyökkäysten muuttuessa yhä hienostuneimmiksi, eivät perinteiset
tietoturvamenetelmät enää riitä suojaamaan loppukäyttäjiä tai palvelun
ylläpitäjiä. Tästä syystä erilaisten tietoturvaratkaisuiden ympärillä
käy kova kuhina, ja aihepiiri on herättänyt suurta kiinnostusta
tutkijoiden keskuudessa. Erilaisia ratkaisuja, joissa on pyritty
selvittämään tietoturvahyökkäyksiä ja näiden mukana tulevia haasteita,
on lukematon määrä. Haaste on löytää ne menetelmät, jotka oikeasti
toimivat riittävällä tarkkuudella ja nopeudella.

Tietoturvahyökkäysten tunnistaminen perustuu siihen, että
analysoimalla yhtä tai useampaa tapahtumaa pyritään löytämään
viitteitä tapahtuvista hyökkäyksistä. Menetelmät jaetaan kahteen eri
tyyppiin: anomalioiden eli poikkeavuuksien tunnistamiseen
(engl. \textit{anomaly detection}) ja väärinkäytösten tunnistamiseen
(engl. \textit{misuse detection}). Näistä anomalioiden tunnistaminen
perustuu malleihin, joita luodaan järjestelmän, käyttäjän tai verkon
normaalista käyttäytymisestä. Tulevaa liikennettä sitten verrataan 
näin luotuihin malleihin, jolloin opituista malleista poikkeava liikenne
voidaan tunnistaa. Väärinkäytösten tunnistamiseen tarkoitetut järjestelmät 
puolestaan sisältävät joukon kuvauksia (engl. \textit{signature}) tunnetuista hyökkäyksistä. 
Tuleva liikenne tarkistetaan näitä kuvauksia vastaan, jolloin näitä vastaavat
hyökkäykset tunnistetaan. Hyökkäysten tunnistusjärjestelmät
jaetaan joissakin tapauksissa myös sen mukaan, mistä tutkittava
liikenne on kerätty. Tällöin järjestelmät on jaettu verkkoon
pohjautuviksi (engl. \textit{network based}) ja asiakaspohjaisiksi
(engl. \textit{host based}).

Tutkielman rakenne on seuraavanlainen. Luvussa 2 esitellään yleisellä
tasolla Web-palvelinalustoja, ja käydään läpi Web-palveluiden
erilaisia toteutustapoja. Luvussa 3 perehdytään uudenlaisiin
Web-ratkaisuihin ja tietoturvahyökkäyksiin, joiden kohteeksi
Web-palvelut nykyisin joutuvat. Luku 6 sisältää katsauksen
tutkimuskenttään liittyvään tutkimukseen. Viimeinen luku sisältää
yhteenvedon tästä työstä.

% -*- mode: LaTeX; coding: utf-8; -*-

\chapter{WWW-palvelun arkkitehtuuri}

WWW-palveluilla tarkoitetaan järjestelmiä, jotka kommunikoivat
keskenään käyttäen vakiintuneita Web-tekniikoita. Web-palvelimet eivät
ole riippuvaisia mistään tietystä laitteisto- tai
käyttöjärjestelmäarkkitehtuurista. WWW-palveluiden käyttämät
teknologiat eivät sinänsä ole erityisen vallankumouksellisia, mutta
tiedonvaihdon helpottuminen standardien protokollien ja WWW:n
hajautetun rakenteen ansiosta on tehnyt WWW-palveluista
mielenkiintoisia niin kehittäjien kuin käyttäjienkin
näkökulmasta\cite{javaweb}.

IBM määrittelee WWW-palvelut seuraavasti: ``Web-palvelut ovat
itsenäisiä ja modulaarisia sovelluksia, jotka voidaan julkistaa,
määritellä, paikallistaa ja suorittaa verkon ylitse, yleeensä WWW:n
välityksellä.''\cite[s.4]{websecurity}

Tietoturvan kannalta WWW-palvelut ovat haastavia, koska niitä
käytetään Internetin välityksellä eivätkä ne ole rajoittuneita
esimerkiksi tietyn organisaation lähiverkkoon. Myös osia
WWW-palveluista sijaitsee usein fyysisesti eri paikoissa ja ne
kommunikoivat keskenään Internetin välityksellä.

TODO.

% -*- mode: LaTeX; coding: utf-8; -*-

\chapter{Web-palveluiden tulevaisuus}

TODO.

% -*- mode: LaTeX; coding: utf-8; -*-

\chapter{Aiheeseen liittyvä tutkimus}

\section{Yleistä}

Viime vuosina web-sovellukset ovat kasvattaneet suuresti suosiota, ja nykyisin yhä useampi palvelu, jossa tietoturvan ja saatavuuden takaaminen on elintärkeää, on siirtynyt osittain tai kokonaan 
verkon puolelle. Tämä on vaikuttanut suuresti siihen, mihin tietoturvahyökkäykset nykyisin kohdistuvat ja kuinka niitä pyritään toteuttamaan. Hyökkäysten muuttuessa yhä hienostuneimmiksi, 
eivät perinteiset tietoturvamenetelmät enää riitä suojaamaan loppukäyttäjiä tai palvelun ylläpitäjiä. Tästä syystä erilaisten tietoturvaratkaisuiden ympärillä käy kova kuhina, ja aihepiiri 
onkin herättänyt suurta kiinnostusta tutkijoiden keskuudessa. Erilaisia ratkaisuja, joissa on pyritty selvittämään aikaisemmissa luvuissa esitettyjä tietoturvahyökkäyksiä ja näiden 
mukana tulleita haasteita, on lukematon määrä. Seuraavaksi esitellyt tutkimukset edustavat vain pientä osaa koko tutkimuskentästä, mutta näistä käy kuitenkin jo ilmi se, millaisiin ratkaisumalleihin
nykyisin pyritään.

Hyökkäysten tunnistaminen toimii siten, että analysoimalla yhtä tai useampaa tapahtumaa pyritään löytämään viitteitä tapahtuvista hyökkäyksistä. Tunnistamiseen pohjautuvat menetelmät jaetaan usein
kahteen eri tyyppiin: anomalioiden eli poikkeavuuksien tunnistamiseen (eng. anomaly detection) ja väärinkäytösten tunnistamiseen (eng. misuse detection). Näistä anomalioiden tunnistamiseen perustuvat
järjestelmät luovat malleja järjestelmän, käyttäjän tai verkon normaalista käyttäytymisestä, ja vertaamalla tapahtumia näin muodostettuihin kuvauksiin voidaan poikkeavuudet tunnistaa. Väärinkäytösten 
tunnistamiseen tarkoitetut järjestelmät taas sisältävät joukon kuvauksia eli signatuureja tunnetuista hyökkäyksistä. Tuleva liikenne tarkistetaan näitä kuvauksia vastaan, jolloin kuvauksia vastaavat
hyökkäykset voidaan tunnistaa. Joissakin tapauksissa jaottelu on myös tehty sen mukaan, mistä tutkittava liikenne on kerätty. Tällöin järjestelmät on jaettu verkkoon pohjautuviksi (eng. network based)
ja asiakaspohjaisiksi (eng. host based). 

\section{Väärinkäytösten tunnistaminen}

Väärinkäytökseen perustuvat järjestelmät ovat pitkään olleet suosituin lähestymistapa hyökkäysten torjumisessa. Nämä järjestelmät on voitu jakaa kahteen erilliseen osaan, jossa tilattomassa 
järjestelmässä jokaista tulevaa tapahtumaa käsitellään itsenäisesti, kun taas tilallisessa järjestelmässä tutkitaan tapahtumien välisiä suhteita. Web-pohjaisten hyökkäysten tutkimisessa tietolähteinä 
on käytetty mm. palvelimien tuottamaa lokia \cite{LightTool} ja IDS-järjestelmien \cite{Snort}\cite{Bro} tapauksessa analysoimalla verkkokerroksen liikennettä. Ensimmäisessä ratkaisussa ongelmana
on, että itse palvelimiin kohdistuvia hyökkäyksiä ei pystytä havaitsemaan ainoastaan lokitietoa tarkkailemalla. Samoin hyökkäyksiä, jotka koostuvat monista eri vaiheesta, on mahdotonta mallintaa. 
Verkkokerroksella toimivia järjestelmiä taas pystytään harhauttamaan muokkaamalla hyökkäyksiä, ja näistä harvat mahdollistavat tilallisen analyysin. 

Esitettyjä ongelmia on pyritty ratkaisemaan lisäämällä havaittavien tapahtumien määrää ja keräämällä informaatiota eri lähteistä. Esimerkiksi asentamalla sovellustasolle erillinen tiedonkeruuseen 
tarkoitettu komponentti \cite{Application} on saatu hyviä tuloksia. Tässä tapauksessa tiedonkeruu tapahtui Apache-palvelimelle asennetun komponentin välityksellä, joka tarkkaile lokitiedon lisäksi
mm. pyyntöjen tulkitsemiseen ja toteuttamiseen menevää aikaa. Tämän ratkaisun etuna on myös se, että tutkittava data on salaamattomassa muodossa ja sessioiden uudelleenrakentaminen on mahdollista. 
Tällä tasolla toimiva IDS-järjestelmä voi myös toimia ennaltaehkäisevästi eli haitalliseksi havaittu liikenne voidaan tiputtaa pois ennen kuin se käsitellään palvelimella. Suurimpana haittapuolena on, 
että tietylle sovellukselle suunniteltua komponenttia ei voida sellaisenaan käyttää muilla alustoilla. 

WebSTAT \cite{Webstat} on tilalliseen analyysiin perustuva IDS-järjestelmä, joka pohjautuu STAT-kehitysympäristöön \cite{STAT}. WebSTAT hyödyntää STATL-ohjelmointikieltä, joka mahdollistaa hyökkäysten
mallintamisen ottamalla huomioon mm. eri tapahtumien välisiä yhteyksiä ja verkkohistoriaa sekä palvelimien kuten Apachen ja Microsoftin IIS:n lokia. Koska tietoa kerätään yhtäaikaisesti monesta eri 
lähteestä, voidaan palvelimien lokitiedon analysointiin yhdistää alempien toimintojen kuten jär\-jes\-tel\-mä- ja verkkotason tuottamaa tietoa. Näin tehty analyysi kuvaa koko järjestelmä tilaa, jolloin 
yllättävät muutokset pystytään havaitsemaan nopeasti. Testien mukaan tällainen analyysi voidaan toteuttaa suurissa järjestelmissä reaaliajassa aiheuttamatta suurempaa viivettä palvelimien toiminnassa. 
Menetelmän sovittaminen tiettyyn järjestelmään vaatii jonkin verran manuaalista työtä, ja tätä voidaankin pitää sen suurimpana heikkoutena. Monimutkaisten hyökkäysten kuvaaminen on usein myös hankala
toteuttaa, ja niiden tulkitsemiseen saattaa tuhlaantua turhaa aikaa.

Väärinkäytösten tunnistamiseen tarkoitettuja järjestelmiä on hyödynnetty myös uudempien tietoturvahyökkäysten tunnistamisessa. Esimerkiksi vihamielisten Flash-mainosten tunnistamiseen tarkoitettu 
OdoSwiff \cite{FlashAdd} pyrkii etsimään sivuilla olevista mainoksista hyökkäykseen tarkoitettua koodia käyttäen staattista ja dynaamista analyysia. Palvelinpuolella XSS-hyökkäyksiä vastaan on luotu 
järjestelmä, josta löytyy yleisimpien hyökkäysten kuvaukset \cite{SignatureXSS}. Kumpikin järjestelmistä toimii erittäin hyvin, kunhan hyökkäys on entuudestaan tuttu.

\section{Poikkeavuuksien tunnistaminen}

Väärinkäytösten tunnistamiseen käytettyjen järjestelmien suurin heikkous piilee siinä, että mallintamattomat hyökkäykset jäävät näiltä huomaamatta. Tämä on erityisesti web-palveluihin kohdistuvien
hyökkäysten tapauksessa iso ongelma, sillä toimintaympäristö muuttuu jatkuvasti, ja uusia hyökkäyksiä ja vanhojena variaatioita ilmestyy tiuhaan tahtiin. Tällöin signatuurien pitäminen ajantasalla
muodostuu mahdottomaksi tehtäväksi. Hyökkäysten monimuotoisuus onkin johtanut siihen, että nykyisin yhä useammassa järjestelmässä pyritään tunnistamaan poikkeavat tapahtumat normaaliin liikenteen seasta
ilman tarkkoja signatuureja. 

Anomalioiden tunnistamiseen on käytössä useita eri menetelmiä, ja ne voidaan jakaa kahteen eri ryhmään \cite{State}. Näistä ensimmäinen sisältää oppimiseen pohjautuvat mallit, jossa normaali 
käyttäytyminen opetetaan opetusmateriaalin avulla. Normaalia käyttäytymistä esittävät profiilit voidaan mallintaa käyttäen joko sään\-tö\-poh\-jais\-ta-, mallipohjaista- tai tilastopohjaista menetelmää. Näistä
sääntöpohjaiset menetelmät muistuttavat eniten perinteisiä IDS-järjestelmiä sillä erolla, että luodut säännöt pohjautuvat kerättävään dataan, ja säännöt voivat olla rakenteiltaan hyvin monimutkaisia. 
Mallipohjaisissa menetelmissä taas luodaan normaalia käyttäytymistä kuvaavat profiilit, jota vastaan tuleva liikenne arvioidaan. Tiedonlouhinta, neuroverkot ja liikenteestä luotujen kuvioiden vertaaminen
tulevaan liikenteeseen (eng. pattern matching) ovat tekniikoita, joita on käytetty tällaisten mallien luomiseen. Analyysissa voidaan esimerkiksi tutkia verkkopakettien kuormia \cite{Payload}\cite{ULISSE} 
ja klusteroimalla ja luokittelemalla liikenne protokollien ja palveluiden mukaan \cite{Cluster}. Viimeisen ryhmän muodostavat tilastollisiin menetelmiin pohjautuvat menetelmät \cite{PacketHeader}, jotka
ovat jääneet vähemmälle huomiolle johtuen alati muuttuvasta toimintaympäristöstä.

Toisen ryhmä anomalioiden tutkimisessa muodostavat spesifistiset mallit (eng. spesification model). Nämä menetelmät pohjautuvat enemmän ihmisten huomioihin ja asiantuntijuuteen kuin matemaattisiin
kuvauksiin. Menetelmissä käytetään useita eri elementtejä aina sovellustasolta verkkotasolle, ja näitä käyttäen luodaan normaalia käyttäytymistä kuvaavat mallit. Järjestelmät voivat hyödyntää esimerkiksi
protokollista kerättävää tietoa anomalioiden tunnistamisessa. Järjestelmien eri tiloja ja tapahtumien välisiä suhteita voidaan myös mallintaa, jolloin poikkeavat tilat ja tapahtumaketjut voidaan
tunnistaa. 

Anomalioiden tunnistamiseen perustuville järjestelmille löytyy useita eri käyttökohteita. Niillä voidaan esimerkiksi pyrkiä tunnistamaan tietokantoihin kohdistuvia tunnettuja ja tuntemattomia SQL-hyökkäyksiä.
Järjestelmälle voidaan opettaa normaali käyttäytyminen esimerkiksi tiedonlouhintamenetelmin \cite{Data} tai luomalla profiileja normaalista käyttäytymisestä \cite{SQLanomaly}\cite{SQLlearning}. Erilaisia
menetelmiä voidaan myös yhdistellä, jolloin todennäköisyys poikkeavan liikenteen tunnistamiseen kasvaa. Tutkimalla esimerkiksi yhtä aikaisesti sekä web-pyynnöissä että SQL-kyselyissä ilmeneviä poikkeavuuksia, 
voidaan tulevat kyselyt pisteyttää tarkasti \cite{WebSQL}. Kyselyiden kategoriointi mahdollistaa sen, että haitalliseksikin merkityt kyselyt voidaan ohjata sellaisille palvelimille, joilla ei ole 
pääsyä arkaluontoiseen tietoon. 

Web-palveluihin kohdistuvien hyökkäysten tunnistaminen on hankala ja aikaa vievä prosessi. Aikaisemmin tässä työssä esitetyt hyökkäykset kattavat vain osan hyökkäyksistä, joita vastaan sovellussuunnittelijat ja
ylläpitäjät joutuvat suojautumaan. Juuri tämä hyökkäysten monimuotoisuus on se seikka, joka on nostanut anomaliatutkimuksen muiden menetelmien yläpuolelle, ja aihepiiri on viime vuosina noussut yhdeksi 
puhutuimmista tietoturvan saralla. 

Poikkeavan tilan tunnistamiseen on käytetty monia eri menetelmiä, ja päätöksen tekemiseen on käytetty useita eri tietolähteitä. Esimerkiksi Swaddler \cite{Swaddler} on 
web-sovelluksille suunniteltu menetelmä, joka oppii kriittisten järjestelmäkutsujen ja sovelluksen tilojen väliset suhteen analysoimalla web-sovelluksen sisäisiä tiloja. Tällä tavoin voidaan tunnistaa hyökkäykset,
jotka aiheuttavat poikkeavia tiloja esimerkiksi sovelluksen normaaliin työkulkuun. Järjestelmä koostuu eri malleista ja osista, joille opetetaan harjoittelujakson aikana sitä vastaava normaali käytös. Jokaiselle
osalle, jotka käytännössä vastaavat tiettyjä sovelluksen toimintoja, lasketaan myös kynnysarvo, jonka ylittyessä sen aiheuttanut toiminto lasketaan anomaliaksi. Tehdyissä testeissä järjestelmä tunnisti 
kaikki toteutetut hyökkäykset, ja virheellisten positiivisten hälystysten määrä oli kohtuullisen pieni. Analysointi aiheutti jonkin verran kuormaa palvelimelle, mutta optimoimalla toteutusta tämä voidaan poistaa
lähes kokonaan. 

Aikaisemmin esitettyä tapaa, jossa hyödynnetään palvelimen tuottamaa logia, voidaan käyttää myös poikkeavuuksien tunnistamisessa \cite{Multi}. Tässä tapauksessa analysoitiin Apachen tuottamaa HTTP-logia, ja huomio kiinnitettiin kyselyihin, joissa parametreja käyttäen välitettiin arvoja palvelinpuolen ohjelmille tai aktivoitiin dokumentteja. Kyselyt purettiin osiin, ja niitä analysoitiin käyttäen useita eri malleja 
mm. kyselyjen pituutta ja normaalia järjestystä, merkkien jakaumia ja pyyntöjen tiheyttä. Vastaavaa mallia on sovellettu myös poikkeavien järjestelmäkutsujen tunnistamisessa \cite{SystemCall}, joten sen käyttö ei rajoitu pelkästään web-palveluihin. Väärinkäytös- ja anomaliamenetelmien yhdistämistä on myös ehdotettu \cite{Combination}. Tällainen järjestelmä voi toimia siten, että tuleva liikenne syötetään ensiksi anomalioita tutkivalle järjestelmälle, ja vain tunnistetut positiiviset tapaukset ohjataan väärinkäytösjärjestelmälle. Menetelmän etuna on, että virheellisten positiivisten määrä tippuu suuresti, ja tarkan analyysin vaativat tapahtumat pienenevät.



% -*- mode: LaTeX; coding: utf-8; -*-

\chapter{Yhteenveto}

Monien perinteisten toimintojen siirtyessä kohti digitaalisia palvelumalleja, on erilaisten 
Web-sovellusten ja -palveluiden merkitys kasvanut huomattavasti. Lisäksi yhä useampi näistä sovelluksista
toimii tietoturvan kannalta kriittisillä sektoreilla.
Tämän muutoksen ovat havainneet myös tietomurtoja tekevät tahot, jotka pyrkivät kaikin
keinoin hyödyntämään järjestelmistä löytyviä heikkouksia rikollisiin tarkoituksiin.

Tietoturva on aina kulkenut alan muuta kehitystä hiukan jäljessä, ja tätä seikkaa hyökkääjät
ovat aina käyttäneet hyväksi. Nykytilanne noudattaa myös samaa kaavaa. Perinteisiltä
tietoturvahyökkäyksiltä osataan jo kohtuullisesti suojautua ja löydetyt haavoittuvuudet
korjataan riittävän nopeasti. Sovellusten tekijät ja palveluiden tarjoavat tuntevat
myös sen verran hyvin oman toimintaympäristönsä, että suuria virheitä ei pääse syntymään.

Web-pohjaisissa palveluissa tilanne on toinen. Aihealue on vielä sen verran tuore, että sovellusten kehittäjät ja ylläpitäjät yrittävät vasta sopeutua siirroksen tuomiin muutoksiin. Uusien teknologioiden ja kehitysympäristöjen oppimiseen menee aina oma aikansa ja näiden turvalliseen käyttöön vielä pidempään.  Tähän kun lisätään mukaan uudenlaiset arkkitehtuurimallit, joihin kehittäjät yrittävät sopeutua, niin on varmaa, että sovelluksiin eksyy erilaisia haavoittuvaisuuksia.

Tähän muutoksen tuomaan epävarmuuteen tietomurtoja ja -varkauksia tekevät hyökkääjät ovat iskeneet. 
Erilaisia uusia hyökkäystapoja kehitetään jatkuvasti ja väärinkäytettyjä haavoittuvaisuuksia löydetään lähes päivittäin.
Monet näistä hyökkäyksistä eivät edes pyri ohittamaan asetettuja suojauksia, vaan ne käyttävät hyödykseen
sovelluksen omia toimintoja, joita ei vain ole osattu riittävästi suojata. Osa näistä heikkouksista on 
myös niin syvällä käytetyssä tekniikoissa tai alustoissa, että suoraan niitä ei pystytä edes korjaamaan vaan
korjaaminen vaatii kokonaan uudenlaista lähestymistapaa. Tämä ajaakin monet suunnittelijoista tekemään
omia pikaisia korjauksia, joiden toimivuus on usein kuitenkin hyvin tapauskohtaista. 

Uudenlaisten tietoturvahyökkäysten tunnistaminen vanhoilla työkaluilla on usein hyvin hankalaa tai
lähes mahdotonta. Yhteistä erilaisille hyökkäyksille on, että ne poikkeavat yleensä jollakin tavoin normaalista
tietomassasta. Ongelmana on se, kuinka nämä poikkeavuudet eli anomaliat tunnistetaan suuresta tietomassasta
riittävällä nopeudella ja tarkkuudella. Tätä tunnistamista varten onkin kehitetty useita erilaisia menetelmiä,
jotka pyrkivät jollakin tavoin erottamaan hyökkäykset muusta liikenteestä. 

Poikkeavuuksien havaitsemiseen perustuvat menetelmät ovat kehittyneet viime vuosina voimakkaasti ja tulevaisuudessa niitä pystyttäneen hyödyntämään myös käytännön sovelluksissa.




\bibliography{lahteet}{}
%\bibliographystyle{finplain}
\bibliographystyle{ieeetr}

%\appendix

%\chapter{Published article}
%
%\includepdf[pages=-]{published_article.pdf}

\end{document}
