% -*- mode: LaTeX; coding: utf-8; -*-

\termlist

\begin{abbrlist}{ANOVA}
\item[AJAX]	Asynchronous JavaScript And XML. Joukko Web-sovelluskehityksessä käytettyjä tekniikoita.
\item[ARP]	Address Resolution Protocol. Ethernet-verkoissa käytettävä protokolla, jolla selvitetään IP-osoitetta vastaava MAC-osoite.
\item[CSRF]	Cross Site Request Forgery. Web-sovelluksissa esiintyvä tietoturva-aukko, joka väärinkäyttää selaimen ja sivuston välistä luottamussuhdetta.
\item[CSV]	Comma-Separated Values. Tiedostomuoto, jolla tallennetaan yksinkertaista taulukkodataa.
\item[DDoS]	Distributed Denial of Service. Hajautettu palvelunestohyökkäys, jossa hyökkäys tapahtuu samanaikaisesti useasta eri kohteesta.
\item[DOM]	Document Object Model. Ohjelmointirajapinta, joka mahdollistaa HTML- ja XHTML-dokumenttien sisällön muokkauksen.
\item[DoS]	Denial of Service. Palvelunestohyökkäys, jolla pyritään lamauttamaan tietty verkkopalvelu.
\item[HTML]	Hypertext Markup Language. Standardoitu kuvauskieli, jolla kuvataan hyperlinkkejä sisältävää tekstiä.
\item[HTTP]     Hypertext Transfer Protocol. Selainten ja Web-palvelimien käyttämä tiedonsiirtoprotokolla.
\item[ICMP]	Internet Control Message Protocol. TCP/IP-pinon kontrolliprotokolla, jolla lähetetään nopeasti viestejä koneelta toiselle.
\item[IDS]	Intrusion Detection System. Järjestelmä, joka on ohjelmoitu tunnistamaan verkkoon suuntautuvat hyökkäysyritykset.
\item[IIS]      Internet Information Services. Microsoftin kehittämä Web-palvelinalusta.
\item[IPS]	Intrusion Prevention System. Järjestelmä, jolla pyritään ennakoivasti estämään tietomurtoyritykset.
\item[IP]	Internet Protocol. Verkkokerroksen protokolla, joka huolehtii tietoliikennepakettien välittämisestä pakettikytkentäisessä Internet-verkossa.
\item[IRC]	Internet Relay Chat. Pikaviestintäpalvelu, joka mahdollistaa reaaliaikaisen keskustelun Internet-käyttäjien välillä.
\item[JSON]	JavaScript Object Notation. JavaScript-ohjelmissa käytetty yksinkertainen tiedonsiirtomuoto.
\item[LDAP]	Lightweight Directory Access Protocol. Hakemistopalveluiden käyttöön tarkoitettu verkkoprotokolla.
\item[MAC]	Media Access Control. Ethernet-verkossa verkkosovittimen yksilöivä osoite.
\item[OSI]	Open Systems Interconnection Reference Model. Eri tiedonsiirtoprotokollista muodostuva kuvaus.
\item[R2L]	Remote-to-Local. Hyökkäystyyppi, jossa hyökkääjä pyrkii saamaan koneelle laajemmat käyttöoikeudet, kuin hänellä muuten olisi.
\item[SOA]	Service Oriented Architecture. Joustava arkkitehtuuriratkaisu, jolla pyritään helpottamaan eri palveluiden välistä kommunikointia.
\item[SQL]	Structured Query Language. Standardoitu kyselykieli, jolla relaatiotietokantaan voi tehdä hakuja, muutoksia ja lisäyksiä.
\item[TCP]	Transmission Control Protocol. Kuljetuskerroksen protokolla, joka rakentuu IP-protokollan päälle. TCP:tä käytetään lukuisissa sovelluskerroksen protokollissa, esimerkiksi HTTP:ssä.
\item[U2R]	User-to-Root. Hyökkäys, jossa hyökkääjä pyrkii hankkimaan koneelle pääkäyttäjän oikeudet.
\item[UDP]	User Datagram Protocol. Yhteyskäytäntö, jolla sovellus voi lähettää viestejä toiselle tietokoneelle.
\item[VPN]	Virtual Private Network. Menetelmä, jolla useampi verkko voidaan yhdistää näennäisesti samaan yksityiseen verkkoon käyttäen julkista verkkoa. 
\item[XHR]	XMLHttpRequest. Ohjelmointirajapinta, jota käytetään esimerkiksi Ja\-va\-Scrip\-tis\-sä.
\item[XHTML]    eXtensible Hypertext Markup Language. HTML:stä kehitetty kuvauskieli, joka täyttää XML:n muotovaatimukset.
\item[XML]	eXtensible Markup Language. Merkintäkieli, jota käytetään sekä formaattina tiedonvälitykseen järjestelmien välillä että dokumenttien tallentamiseen.
\item[XSS]	Cross Site Scripting. Web-sovelluksissa esiintyvä tietoturva-aukko, joka mahdollistaa käyttäjän koodin syöttämisen sovellukselle.  
\item[Xpath]    XML Path Language. XML-dokumenttien osien osoittamiseen tarkoitettu kieli.	
\end{abbrlist}
