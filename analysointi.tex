% -*- mode: LaTeX; coding: utf-8; -*-

\chapter{Tiedon analysointi}

TODO.

\section{Tiedonkäsittely-ympäristön asentaminen}

Analysoinnissa käytettävän MySQL:n versio tulisi olla vähintään
5.1. Tätä edeltävissä versiossa on rajoitus, joka estää yli 2 gigatavun
kokoisen pakatun tietokannan taulun muodostamisen \cite{archive2g}.

\section{Analysoinnin vaiheet}

Analysointi koostuu kolmesta vaiheesta. Ensimmäiseen kuuluu tiedon
kerääminen tutkittavasta palvelusta. Tutkittaessa WWW-palvelimen
käyttöä riittää usein kerätä vierailijalokeja analyysiä
varten. Esimerkiksi Apachen ``combined access log'' soveltuu tähän
tarkoitukseen hyvin.

Toisessa vaiheessa tekstimuotoiset lokitiedostot luetaan ja
käsitellään koneellisesti helpommin analysoitavaan
muotoon. Tätä työtä varten kehitetyssä tiedonkäsittelijässä
lokitiedoston eri kentät palastellaan ja tieto tallennetaan
relaatiotietokantaan. Lopuksi eri kenttien numeeriset ja
luokka-asteikolliset arvot klusteroidaan.

Kolmannessa vaiheessa tapahtuu varsinainen analysointi. Käytetty
anomalia-analyysi edellyttää, että aineiston muuttujat ovat
luokka-asteikollisia ja tästä johtuen tieto on klusteroitu edeltävässä vaiheessa.

\subsection{Tiedon keruu}

TODO.

\subsection{Esikäsittely}

Sopivaa yksivaiheista parseria käyttämällä olisi mahdollista käsitellä
lähtödata suoraan analyysissä käytettävään muotoon. Käytännössä kuitenkin datan
esikäsittely kannattaa hoitaa useammassa vaiheessa, jotta datassa
olevat puuttuvat tai poikkeavat arvot tulee huomioitua
asianmukaisesti. Monivaiheinen esikäsittely helpottaa myös saman
lähtödatan käyttämisen useaan eri analyysiin.

Tiedon käsittelyn helpottamiseksi tässä työssä käytetään esikäsittelyn
välivaiheiden ja lopputuloksen tallentamiseen
relaatiotietokantaa. Relaatiotietokanta mahdollistaa useiden
esikäsittelyn vaiheiden suorittamisen vähäisellä ohjelmoinnilla ja
helposti ymmärrettävästi.

Tässä työssä käsiteltävää aineistoa varten on kehitetty ``PhasefulSplitter
'' -sovellus esikäsittelyä varten.

TODO.

\subsection{Kategorisointi}

TODO.

\subsection{Analyysi}}

TODO. Tuomon ja Juhon maaginen anomaliamylly tähän.
