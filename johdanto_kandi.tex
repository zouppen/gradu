% -*- mode: LaTeX; coding: utf-8; -*-

\chapter{Johdanto}

Viime vuosina Web-sovellusten ja -palveluiden suosio on 
ollut voimakkaassa kasvussa, ja monet palveluista ovat nykyisin kriittisiä osia
yhteiskuntamme toimivuuden kannalta. Internetin välityksellä
käytettävien palveluiden tietoturvan ja saatavuuden takaaminen on
tämän johdosta noussut monessa yrityksessä tärkeään rooliin.
Tietojärjestelmien siirtyminen julkishallinnon ja
yritysten erillisverkoista julkiseen Internetiin on vaikuttanut omalta
osaltaan myös siihen, mihin tietoturvahyökkäykset nykyisin
kohdistuvat ja kuinka ne pyritään toteuttamaan.

Toimivan ja turvallisen Web-pohjaisen palvelun tarjoaminen vaatii
nykyisin todella paljon aikaa ja huolellisuutta sekä palvelun
kehittäjältä että palvelun tarjoajalta ja ylläpitäjältä. Lisäksi
siirtyminen palvelukeskeiseen arkkitehtuuriin on johtanut siihen, että
palvelut ovat hajautuneet useammalle palvelimelle jopa eri puolella
maapalloa. Arkkitehtuurin muutoksen myötä myös tietomurtojen
tekijöiden kohteeksi on avautunut uusia rajapintoja.

Hyökkäysten muuttuessa yhä hienostuneimmiksi, eivät perinteiset
tietoturvamenetelmät enää riitä suojaamaan loppukäyttäjiä tai palvelun
ylläpitäjiä. Tästä syystä erilaisten tietoturvaratkaisuiden ympärillä
käy kova kuhina, ja aihepiiri on herättänyt suurta kiinnostusta
tutkijoiden keskuudessa. Erilaisia ratkaisuja, joissa on pyritty
selvittämään tietoturvahyökkäyksiä ja näiden mukana tulevia haasteita,
on lukematon määrä. Haaste on löytää ne menetelmät, jotka oikeasti
toimivat riittävällä tarkkuudella ja nopeudella.

Tutkielman rakenne on seuraavanlainen. Luvussa 2 esitellään yleisellä
tasolla Web-palvelinalustoja, ja käydään läpi Web-palveluiden
erilaisia toteutustapoja. Luvussa 3 perehdytään uudenlaisiin
Web-ratkaisuihin ja tietoturvahyökkäyksiin, joiden kohteeksi
Web-palvelut nykyisin joutuvat. Luku 6 sisältää katsauksen
tutkimuskenttään liittyvään tutkimukseen. Viimeinen luku sisältää
yhteenvedon tästä työstä.
