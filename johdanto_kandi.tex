% -*- mode: LaTeX; coding: utf-8; -*-

\chapter{Johdanto}

Viime vuosina Web-sovellusten ja -palveluiden suosio on 
ollut voimakkaassa kasvussa, ja monet palveluista ovat nykyisin kriittisiä osia
yhteiskuntamme toimivuuden kannalta. Internetin välityksellä
käytettävien palveluiden tietoturvan ja saatavuuden takaaminen on
tämän johdosta noussut monessa yrityksessä tärkeään rooliin.
Tietojärjestelmien siirtyminen julkishallinnon ja
yritysten erillisverkoista julkiseen Internetiin on vaikuttanut omalta
osaltaan myös siihen, mihin tietoturvahyökkäykset nykyisin
kohdistuvat ja kuinka ne pyritään toteuttamaan.

Hyökkäysten muuttuessa yhä hienostuneimmiksi, eivät perinteiset
tietoturvamenetelmät enää riitä suojaamaan loppukäyttäjiä tai palvelun
ylläpitäjiä. Tästä syystä erilaisten tietoturvaratkaisuiden ympärillä
käy kova kuhina, ja aihepiiri on herättänyt suurta kiinnostusta
tutkijoiden keskuudessa. Erilaisia ratkaisuja, joissa on pyritty
selvittämään tietoturvahyökkäyksiä ja näiden mukana tulevia haasteita,
on lukematon määrä. Haaste on löytää ne menetelmät, jotka oikeasti
toimivat riittävällä tarkkuudella ja nopeudella.

Tietoturvahyökkäysten tunnistaminen perustuu siihen, että
analysoimalla yhtä tai useampaa tapahtumaa pyritään löytämään
viitteitä tapahtuvista hyökkäyksistä. Menetelmät jaetaan kahteen eri
tyyppiin: anomalioiden eli poikkeavuuksien tunnistamiseen
(engl. \textit{anomaly detection}) ja väärinkäytösten tunnistamiseen
(engl. \textit{misuse detection}). Näistä anomalioiden tunnistaminen
perustuu malleihin, joita luodaan järjestelmän, käyttäjän tai verkon
normaalista käyttäytymisestä. Tulevaa liikennettä sitten verrataan 
näin luotuihin malleihin, jolloin opituista malleista poikkeava liikenne
voidaan tunnistaa. Väärinkäytösten tunnistamiseen tarkoitetut järjestelmät 
puolestaan sisältävät joukon kuvauksia (engl. \textit{signature}) tunnetuista hyökkäyksistä. 
Tuleva liikenne tarkistetaan näitä kuvauksia vastaan, jolloin näitä vastaavat
hyökkäykset tunnistetaan. Hyökkäysten tunnistusjärjestelmät
jaetaan joissakin tapauksissa myös sen mukaan, mistä tutkittava
liikenne on kerätty. Tällöin järjestelmät on jaettu verkkoon
pohjautuviksi (engl. \textit{network based}) ja asiakaspohjaisiksi
(engl. \textit{host based}).

Tutkielman rakenne on seuraavanlainen. Luvussa 2 esitellään yleisellä
tasolla Web-palvelinalustoja, ja käydään läpi Web-palveluiden
erilaisia toteutustapoja. Luvussa 3 perehdytään uudenlaisiin
Web-ratkaisuihin ja tietoturvahyökkäyksiin, joiden kohteeksi
Web-palvelut nykyisin joutuvat. Luku 6 sisältää katsauksen
tutkimuskenttään liittyvään tutkimukseen. Viimeinen luku sisältää
yhteenvedon tästä työstä.
